%! TEX root = ../main.tex
\documentclass[main]{subfiles}


\begin{document}
\chapter{関連研究と本研究の位置づけ}
本研究は,農業移動ロボットによる自律巡回を前提として,
高信頼な自己位置推定と生育環境センシングを統合し,圃場空間マップを生成することを目的とする.
本章では,自己位置推定およびセンシングに関する関連研究を整理し,本研究の位置づけを示す.


\section{農業環境における自己位置推定}
\subsection{LiDARを用いた自己位置推定}
自律移動ロボットの自己位置推定において,GNSSが利用できない,
あるいは不安定な環境下では,3D-LiDARとIMUを組み合わせたLiDAR-Inertial Odometry(LIO)が主流となっている.
特に,LOAM (Lidar Odometry and Mapping in Real-time) \cite{ref:loam} の登場以降,
LeGO-LOAM \cite{ref:lego_loam} や LIO-SAM \cite{ref:lio_sam} といったグラフ最適化ベースの手法が開発され,
近年では直接法とカルマンフィルタを組み合わせたFAST-LIO2 \cite{ref:fast_lio2} が,
計算負荷の低さとロバスト性の高さから注目されている.
これらの手法は,一般的な屋外環境や市街地では高精度な推定が可能であるが,
農業用ハウスや圃場のような半構造化環境では課題が残る.
特に,ハウス内の長い通路のような環境では,進行方向に対する幾何学的特徴が欠落するため,
LiDARスキャンマッチングの解が不定となる縮退問題が発生しやすく,
長時間走行における位置ドリフトが不可避となる\cite{ref:greenhouse_lio}.


\subsection{GNSS/INS/LiDARのセンサフュージョン}
LIOのドリフトを抑制するため,GNSSによる絶対位置情報を統合する手法(GNSS/INS/LiDAR Fusion)が広く研究されている.
一般に,拡張カルマンフィルタ(以下,EKF)やグラフ最適化を用いてGNSS観測を拘束条件として追加するが,
既存手法の多くは,GNSSの観測誤差がガウス分布に従うことや,測位状態が安定していることを
前提としている\cite{ref:gnss_fusion_review}.
しかし,農業用ハウス環境では,ビニールやガラス,
金属フレームによる信号遮蔽やマルチパスが頻発し,
測位品質が突発的に劣化する.
不正確なGNSS観測値を不用意に統合すれば,推定軌跡に位置飛びが生じ,
マッピングの整合性が破壊される危険性がある.
したがって,農業環境特有のGNSS品質変動に適応的な,より高度なセンサフュージョンが必要とされている.


\section{農業用生育環境センシングセンサと分光計測}
生育環境情報を圃場空間へ投影してマップ化するためには,
各時刻のセンサ観測に対応する自己位置の誤差がマップ品質を直接規定する.
したがって,自己位置推定の信頼性向上は,
高分解能なセンシング結果を空間マップとして整合的に統合する上で不可欠である.

精密農業において,圃場内の微気象(温度,湿度,CO$_2$濃度など)を
把握することは重要である.
固定式のIoTセンサノードを用いたモニタリングは広く行われているが,
設置コストや電源確保の観点から空間分解能に限界がある.
これに対し,移動ロボットにセンサを搭載し,巡回計測を行うことで,
高密度な環境マップを作成する試みがなされている\cite{ref:mobile_sensing_review}.

\subsection{小型分光センサの農業利用}
近年,作物の生理状態を非破壊で診断するために,
分光反射率の計測が注目されている.
従来の分光器は大型かつ高価であったが,MEMS技術を用いた超小型分光センサ
(浜松ホトニクス製 C12880MA等)が登場し,
ドローンやAGVへの搭載が可能となった\cite{ref:drone_spectral}.


本研究室の先行研究\cite{ref:Kobayashi2021AGV}では,C12880MAを搭載したAGV計測システムが構築されたが,
以下の2つの課題が残されていた.
第一に,当時はAGVの自律走行システムが十分に確立されていなかったため,
移動しながらの連続計測ではなく,特定地点に停止してデータを取得するに留まった点である.
第二に,データ取得速度の不足である.
先行研究の手法は,LPC1768 MCUと外部ADコンバータを用い,
データ取得のトリガ信号ごとにCPUが割り込みサービスルーチン(以下,ISR)を実行する
割り込み駆動方式を採用していた.この方式では,ISR 実行に伴うオーバーヘッドが支配的となり,
サンプリングレートは最大でも \SI{50}{\kilo\hertz} に制限されていた.
一方で,C12880MA 自体は数 \SI{}{\mega\hertz} オーダでの駆動能力を有しており,
観測された性能制限はセンサ固有の制約ではなく,
読み出し回路系の設計に起因するものであった.

\subsection{浜松ホトニクス社製評価基板の課題}
C12880MAの開発・評価用として,浜松ホトニクス社からは評価回路基板C13016が提供されている\cite{ref:hamamatsu_c13016,ref:C12880MA}.
この評価基板は,PCとUSB接続し,専用ソフトウェアを用いてスペクトル波形を確認・保存することができる.
しかし,以下の点において,本研究のような組み込みロボットシステムへの統合には不向きである.
\begin{enumerate}
  \item \textbf{コストとサイズ}: 評価基板単体で高価(約20万円程度)であり,AGVへの多数搭載や低コスト化の障壁となる.
  \item \textbf{インターフェースの制約}: 基本的にPC(Windows)ベースでの動作を前提としており,マイコンから直接かつ高速に制御するためのGPIO/SPI等のインターフェースが公開・最適化されていない.
\end{enumerate}
したがって,AGVの走行速度に合わせて高密度な分光マッピングを行うためには,
C12880MAをMCUから直接,かつ限界性能で駆動できる専用ドライバの開発が不可欠である.


\section{本研究の位置づけと新規性}
以上の背景を踏まえ,本研究の新規性は以下の3点に集約される.
\begin{enumerate}
  \item \textbf{農業環境特有のGNSS品質変動のモデル化}: 
  ハウス環境におけるGNSS測位モード(Fix/Float/No-Fix)の遷移特性と,マルチパスによる外れ値の発生傾向を実データに基づき分析し,センサ融合に必要な統計的特性を明らかにする点.
  \item \textbf{GNSS Supervisorによるロバストな統合}: 
  従来の単純な閾値処理ではなく,時系列の整合性検定(NIS検定)とヒステリシス制御を導入した「GNSS Supervisor」を提案し,測位品質が激しく変動する状況下でも,LIOの軌跡を破綻させずに絶対位置補正を行う点.
  \item \textbf{高速分光センシングと空間マッピングの実証}: 
  C12880MAの性能を最大限に引き出す「デュアルスイッチFSM」駆動方式を独自に開発し,
  従来のISRベース手法に対して二桁以上高速な 5\,MHz クラスのサンプリングを達成するとともに,
  高精度な自己位置推定結果と統合することで,実用的な分解能を持つ生育環境マップの生成を実証する点.
\end{enumerate}

\end{document}