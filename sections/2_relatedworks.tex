%! TEX root = ../main.tex
\documentclass[main]{subfiles}
\begin{document}
\chapter{関連研究}
本研究は,農業移動ロボットによる自律巡回を前提として,
高信頼な自己位置推定と生育環境センシングを統合し,圃場空間マップを生成することを目的とする.
本章では,圃場巡回システム,自己位置推定,GNSSのNLOSと完全性監視,
および農業用センシングに関する関連研究を整理し,本研究の位置づけを明確化する.

\section{圃場巡回システム}
本研究室の先行研究では,AGVを用いた桑畑巡回システムが提案されている\cite{ref:Kobayashi2021AGV}.
同研究は,圃場での巡回観測の有効性を示した一方で,自律走行は主としてシミュレーション上で検証され,
実機における安定運用の観点では課題が残っていた.
また,位置情報としてGNSSを利用していたものの,RTKによる補正を前提としていなかった.

これに対し筆者は,AGVに搭載した2D-LiDAR,IMU,ホイルオドメトリを用い,
圃場の短距離区間における自動走行を実機で実現した\cite{ref:xu2024development}.
しかし,2D-LiDARベースのSLAMは圃場・ハウス環境において観測幾何が単純化しやすく,
自己位置推定の安定性に限界がある.加えて,2D計測では障害物の高さ情報を直接取得できず,
通路脇の枝葉や棚構造など三次元的干渉を伴う状況での安全性評価が困難である.
またホイルオドメトリは,空転・スリップにより誤差が累積しやすい.
以上より,本研究では3D-LiDARとIMUに基づく自己位置推定の主軸とし,
RTK-GNSSを絶対位置拘束として統合するセンサフュージョンを採用することで,
圃場・ハウスでの巡回に必要なロバスト性を確保する.

\section{農業環境における自己位置推定}
\subsection{LiDARを用いた自己位置推定}
GNSSが利用できない,あるいは不安定な環境下では,3D-LiDARとIMUを組み合わせた
LiDAR--Inertial Odometry(以下,LIO)が自己位置推定の中核となる.
LOAM \cite{ref:loam} 以降,LeGO-LOAM \cite{ref:lego_loam},LIO-SAM \cite{ref:lio_sam} などの
最適化ベース手法が発展し,近年では直接法とフィルタ推定を組み合わせたFAST-LIO2 \cite{ref:fast_lio2} が,
計算効率とロバスト性の観点から広く利用されている.

一方,農業用ハウスや圃場のような半構造環境では課題が残る.
特に,長い通路では,進行方向に対して有効な幾何特徴が不足し,
スキャンマッチングが不定となる縮退が発生しやすい.
その結果,長時間走行におけるドリフトが蓄積し,
地図の整合性が低下することが報告されている\cite{ref:greenhouse_lio}.
したがって,農業環境での実運用には,LIO単独に依存せず,外部基準によるドリフト抑制が重要となる.

\subsection{GNSS/INS/LiDARのセンサフュージョンと課題}
LIOのドリフトを抑制するため,GNSSによる絶対位置情報を統合する手法が広く研究されている.
一般に,拡張カルマンフィルタやグラフ最適化によりGNSS観測を拘束として導入するが,
多くの手法はGNSS誤差が概ねガウス的であること,
および受信機が報告する測位状態が信頼できることを暗黙に前提としている\cite{ref:gnss_fusion_review}.

しかし農業用ハウス環境では,ガラス・金属フレーム等に起因する遮蔽・反射により,
NLOS受信やマルチパスが発生し,見かけ上の受信機状態が良好であっても
数メートルのバイアス外れ値が混入し得る.
このような外れ値をバックエンドへ入力すると,推定軌跡に位置飛びが生じ,
ループ閉合や因子グラフ最適化を介して地図全体が破壊される危険性がある.
したがって,GNSSを使う/使わないの二択ではなく,
GNSS拘束を動的に選別し,安全に統合するための品質監視の枠組みが必要となる.

\subsection{NLOS/マルチパスと完全性監視}
GNSS外れ値の典型要因として,マルチパスおよびNLOS受信が挙げられる.
マルチパスは直達波と反射波の干渉により誤差が正負に振れ得る一方,
NLOSでは直達波が遮蔽され反射経路のみを受信するため,測距誤差が系統的バイアスとなりやすい.
このため,受信機が出力する状態量のみでは外れ値を十分に判別できない場合がある.

完全性監視は,測位情報が要求性能を満たさない場合に
利用者へ適時に警告・排除判断を提供する枠組みであり,航空分野でRAIM等として体系化されてきた.
一方,地上移動体では衛星可視数の変動や環境起因の多重劣化が顕著であり,
GNSS内部冗長性に基づく検定のみでは限界がある.
この背景から近年は,深層学習を用いて GNSS の擬似距離誤差をモデル化し,
LiDAR データと融合させる DeepPCO \cite{deepPCO2022}のような
エンドツーエンドの手法も提案されている.
しかし,これらの手法は膨大な学習データを必要とする.

また,SLAMやマルチセンサ融合においては,外れ値拘束が少数混入しただけでも,
推定解が大きく歪み,地図整合性が破綻し得る.
この問題に対し,スイッチ変数を用いて拘束を自動的に無効化する手法\cite{6385590}が提案されている.
これは最適化内部でのロバスト化により外れ値影響を抑える枠組みである.


\subsection{本研究の特徴}
以上の研究動向に対し,本研究は圃場環境でも一部区間で利用可能なRTK-GNSSを活用しつつ,
NLOS/マルチパスに起因する外れ値がバックエンドへ注入されるリスクを,
LIOを短時間の基準として用いた整合性検定により抑制する点に特徴がある.
すなわち,最適化内部のロバスト関数に全面的に依存するのではなく,
因子投入前の段階でGNSS拘束を選別することで,地図破壊の発生確率を低減する.


\section{農業用生育環境センシングセンサと分光計測}
生育環境情報を圃場空間上に配置してマップ化するためには,
各時刻の観測に対応する自己位置の誤差がマップ品質を直接規定する.
したがって,自己位置推定の信頼性向上は,
高分解能なセンシング結果を空間マップとして整合的に統合する上で不可欠である.

精密農業において,圃場内の微気象(温度,湿度,CO$_2$濃度など)を把握することは重要である.
固定式IoTセンサノードは広く用いられているが,設置コストや電源確保の観点から空間分解能に限界がある.
これに対し,移動ロボットにセンサを搭載し巡回計測することで,
高密度な環境マップを作成する試みが報告されている\cite{ref:mobile_sensing_review}.

\subsection{小型分光センサの農業利用}
作物の生理状態を非破壊で診断する手段として,分光反射率の計測が注目されている.
近年,MEMS技術により小型分光センサ(浜松ホトニクス C12880MA等)が普及し,
ドローンやAGVへの搭載を前提とした計測システムが提案されている\cite{ref:drone_spectral}.

先行研究\cite{ref:Kobayashi2021AGV}では,C12880MAを搭載したAGV計測システムが構築されたが,
(1) 移動しながらの連続計測が限定的であった点,
(2) 読み出し回路の設計に起因してデータ取得速度が最大 \SI{50}{\kilo\hertz} 程度に制約された点,
の2点が課題として残されていた.
C12880MA自体は \SI{}{\mega\hertz} オーダの駆動能力を有するため,
観測された性能制限はセンサ固有の制約ではなく,読み出しアーキテクチャの制約に起因すると解釈できる.

\subsection{浜松ホトニクス社製評価基板の課題}
C12880MAの開発・評価用として,浜松ホトニクス社から評価基板C13016が提供されている\cite{ref:hamamatsu_c13016,ref:C12880MA}.
同基板はPCとUSB接続し,専用ソフトウェアによりスペクトル波形の確認・保存が可能である.
しかし,(1) 高コスト・大型であり多数搭載や低コスト化の障壁となる点,
(2) PC前提であり,組み込みロボットで必要となるMCUからの直接・高速制御に最適化されていない点,
から,本研究のようなAGV統合には不向きである.
したがって,AGV走行速度に整合した高密度分光マッピングを実現するには,
C12880MAをMCUから直接,かつ限界性能で駆動できる専用ドライバの開発が不可欠である.

\section{本研究の位置づけと新規性}
以上の背景を踏まえ,本研究の新規性は以下の3点に集約される.
\begin{enumerate}
  \item \textbf{農業環境特有のGNSS品質変動の実データ分析}:
  ハウス環境における測位モードの遷移と,
  NLOS/マルチパスに起因する外れ値の発生傾向を実データに基づき整理し,
  センサフュージョンで問題となる誤拘束注入の条件を明確化する点.
  \item \textbf{GNSS 品質監視による安全な統合}:
  受信機状態量に依存した単純な閾値処理ではなく,
  LIOとGNSSの短時間増分の整合性に基づく検定とヒステリシス制御を組み合わせた
  GNSS品質監視モジュールを提案し,
  測位品質が突発的に劣化する状況下でも地図破壊を回避しつつ絶対位置補正を行う点.
  \item \textbf{高速分光センシングと空間マッピングの統合実証}:
  C12880MAの性能を最大限に引き出す駆動方式を開発し,
  従来のISRベース手法に対して高速な \SI{5}{\mega\hertz} のサンプリングを達成する.
  さらに,高信頼な自己位置推定と統合する点.
\end{enumerate}

\end{document}