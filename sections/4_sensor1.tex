%! TEX root = ../main.tex
\documentclass[main]{subfiles}

\begin{document}

\chapter{高速分光センシングシステムの構築}
\label{ch:sensor_unit}
本章では,農業ロボットによる植生状態観察を目的として構築した
高速分光センシングシステムについて述べる.
対象とする農業環境(信州大学繊維学部の圃場)では,走行しながら連続的に計測を行うため,
分光データの取得は高い時間分解能と安定性が要求される.
一方,本研究で使用する浜松ホトニクス製ミニ分光器C12880MAは,外部クロックに同期して出力が更新されるため,
割込み駆動に基づく一般的な取得方式では高周波領域で取得欠落やタイミングずれが生じやすい.
そこで本研究では,割込み駆動方式の限界を再現的に示した上で,
タイマを用いてC12880MAのCLKを連続生成し,TRGに同期したハードウェアトリガでADC変換を駆動し,
DMAにより画素列をストリーミング取得するアーキテクチャを設計・実装した.
さらに,ST立下り後のソフトウェア起動遅延を実測により定量化し,有効画素ウィンドウの補正として取り込むことで,
\SI{5}{\mega\hertz}級の安定取得を実現した.
提案手法はSTM32F446RE(F4シリーズ)およびSTM32H723ZG(H7シリーズ)に実装し,
最終的に\SI{5}{\mega\hertz}での連続取得を実証した.

\section{本章で用いるセンサユニットの概要}
\label{sec:sensor_unit_overview}
本研究で用いるセンサユニットは,植生状態観察のための分光計測を中核に,
生育環境の基礎量(温度・湿度・気圧,\(\mathrm{CO_2}\)濃度)を同時取得できるよう構成した.
ユニット全体の構成をFig.~\ref{fig:agv_multisensor_overall}に示す.
なお,環境量センサを含むユニットの基本構成および計測ロジックは先行研究で確立されており,
本章ではC12880MAの高速取得に直接関与する要素にに焦点を当てる.
各センサおよびMCUのインタフェース回路は付録に示す
(Fig.~\ref{fig:mcu_circuit},Fig.~\ref{fig:c12880ma_circuit},
Fig.~\ref{fig:s300l_circuit},Fig.~\ref{fig:bme280_circuit}).

\begin{figure}[t]
  \centering
  \includegraphics[width=\linewidth]{figures/4/AGV_Multisensor.pdf}
  \caption{Overall multisensor system architecture}
  \label{fig:agv_multisensor_overall}
\end{figure}


\subsection{構成要素と主仕様}
\label{subsec:sensor_unit_specs}
センサユニットの構成要素と主仕様をTable~\ref{tab:sensor_unit_specs}に示す.
また,開発したセンサユニットの外観をFig.~\ref{fig:sensor_unit_photo}に示す.
C12880MAは外部クロックに同期してVIDEO信号が更新されるラインセンサであり,
MCU側ではトリガ生成とサンプリングタイミングの厳密な同期が必要となる.
BME280およびS300L-3Vは比較的低速な環境量計測であるが,
分光フレームと同一の時刻系で統合することで走行中のマルチモーダル計測を可能とする.

\begin{figure}[t]
  \centering
  \includegraphics[width=\linewidth]{figures/4/placeholder_sensor_unit_photo.pdf}
  \caption{Photographs of the developed sensor unit. 
  (A) Top view of the sensor board integrating the C12880MA, BME280, and S300L-3V. 
  (B) Oblique view showing the component layout. 
  (C) Top view with the optical bandpass filter holder mounted. 
  (D) The fully assembled unit ready for installation.}
  \label{fig:sensor_unit_photo}
\end{figure}


\begin{table}[t]
  \centering
  \caption{Sensor unit components and key specifications (overview)}
  \label{tab:sensor_unit_specs}
  \setlength{\tabcolsep}{4pt}
  \renewcommand{\arraystretch}{1.15}
  \begin{tabular}{p{0.32\linewidth} p{0.22\linewidth} p{0.18\linewidth} p{0.22\linewidth}}
    \hline
    Component & Manufacturer & Interface & Note \\
    \hline \hline
    Mini-spectrometer C12880MA
      & Hamamatsu Photonics
      & Analog (VIDEO), CLK/ST/TRG
      & High-speed synchronized acquisition \cite{ref:C12880MA} \\
    BME280 (AE-BME280)
      & Bosch Sensortec / Akizuki Denshi
      & I\(^2\)C
      & Temperature / humidity / pressure \\
    S300L-3V \(\mathrm{CO_2}\) sensor
      & ELT SENSOR
      & I\(^2\)C
      & \(\mathrm{CO_2}\) concentration \\
    MCU
      & STMicroelectronics
      & ---
      & STM32F446RE (F4) / STM32H723ZG (H7) \\
    \hline
  \end{tabular}
\end{table}

\section{分光センサ駆動モジュールの開発における従来の課題}
C12880MAセンサは,入力ST信号立下り後にTRG信号を出力し,
所定回数のTRGに同期してVIDEO信号が有効となる.
そのタイミング概要をFig.~\ref{fig:c12880ma_timing}に示す.
従来は,TRG信号の立ち上がりで割り込みを発生させ,
割り込みサービスルーチン(以下,ISR)内でADC起動・読出を行う方式が用いられてきた.
本研究室の先行研究では,LPC1768 MCUと外部ADコンバータ(SPI接続)を用い,この方式でC12880MAから
\SI{50}{\kilo\hertz}でのデータ取得が報告されている\cite{ref:Kobayashi2021AGV}.
しかし数\si{\mega\hertz}帯では,\(\mathcal{O}(100\,\mathrm{ns})\)オーダの周期でイベントが到来するため,
ソフトウェア介在の起動遅延が無視できず,ISR処理時間が主要な性能制約となる.
したがって,センサ性能を引き出すにはソフトウェア割込みに依存しないハードウェア同期機構が必要である.

\begin{figure}[ht]
  \centering
  \includegraphics[width=0.9\linewidth]{figures/4/placeholder_C12880MA_timing.png}
  \caption{Timing diagram of the C12880MA: excerpted from the datasheet\cite{ref:C12880MA}}
  \label{fig:c12880ma_timing}
\end{figure}

\begin{figure}[t]
  \centering
  \includegraphics[width=0.98\linewidth]{figures/4/c12880ma_pixel_index_mapping.pdf}
  \caption{C12880MA readout rule and pixel index mapping between sensor-side definition and DMA buffer in this work. 
  The sensor pixel index is defined from the falling edge of ST (\(t=0\)), where pixels \#1--\#88 are dark/invalid and pixels \#89--\#376 are valid (288 pixels). 
  Due to software start latency, sensor pixels \#1--\#4 are not captured; therefore, the nominal valid start at sensor pixel \#89 appears at DMA buffer sample \#85.}
  \label{fig:c12880ma_readout_rule}
\end{figure}

Fig.~\ref{fig:c12880ma_readout_rule}に,本研究で用いる画素番号の定義と,
センサ側インデックスとDMAバッファ上のサンプル番号の対応関係を示す.
本研究ではST信号の立下りを\(t=0\)とし,ST立下り後の第1画素を\#1として画素番号を定義する.
データシートの読出規則では,\#1--\#88は暗画素(無効画素)であり,\#89--\#376が有効画素(288画素)である.
一方,実装ではST立下り後にADC/DMAをソフトウェアで開始するため起動遅延が生じ,
センサ側の\#1--\#4に相当するサンプルが取得できないことを実測により確認した.
その結果,有効画素開始(センサ側\#89)はDMAバッファ上では\#85として観測されるため,
本研究ではDMAバッファの\#85--\#372(288サンプル)を有効分光データとして採用する.
なお,EOS信号は読出終端タイミングの検証に用い,現実装ではDMA転送は固定長(\(N=387\))で終了する.

\subsection{従来方式の再検証}
\label{subsec:reconfirm_limit}
本研究では,内部ADCが比較的高速なSTM32F446REを用いて従来方式の制約を再検証した.
結果をTable~\ref{tab:previous_method_limit}に示す.
割込みのみでは\SI{25.4}{\kilo\hertz}付近で欠落が発生し,
DMAを併用してCPU負荷を低減した場合でも\SI{130}{\kilo\hertz}程度が限界であった.
以上より,従来方式の主要な性能制約が割込み起動遅延およびISR処理時間に起因することを確認した.


\begin{table}[t]
    \centering
    \caption{Limits of acquisition frequency with conventional methods}
    \label{tab:previous_method_limit}
    \begin{tabular}{l|l|c}
        \hline
        \textbf{Method} & \textbf{MCU} & \textbf{Achieved frequency} \\
        \hline \hline
        Interrupt only      & STM32F446RE & \SI{25.4}{\kilo\hertz}\\
        Interrupt + DMA     & STM32F446RE & \SI{130}{\kilo\hertz} \\
        \hline
    \end{tabular}
\end{table}

\section{提案手法:連続CLK下でのST制御とTRG同期ハードウェア取得}
\label{sec:proposed_chain}
本研究では,CLKを停止・再開することなく連続供給し,
ST信号により積分区間を定義した上で,ST立下り後に読出を開始するC12880MAの基本動作に整合する形で取得系を構成した.
\SI{5}{\mega\hertz}級の周期イベントに対してソフトウェア割込み処理を介在させると,
起動遅延および処理時間ばらつき(jitter)が不可避となるため,
本研究ではTRG同期のハードウェアトリガを用いてADC変換を駆動し,
DMAにより画素列をストリーミング転送することで,サンプリング位相の決定性を確保する.

\subsection{フレーム取得シーケンス(ST制御とDMAストリーミング)}
\label{subsec:acq_sequence}
1フレームの取得は次の手順で構成される.
(i) STをHighに設定して積分を開始し,所定の積分時間\(T_{\mathrm{int}}\)を保持する.
(ii) STをLowへ遷移させて積分を終了し,直後にADCをDMAモードで開始する.
(iii) 以降はTRG立上りに同期してADC変換が自動的に実行され,DMAによりバッファへ連続転送される.
(iv) DMA転送完了コールバックによりフレーム完了を検知し,次フレーム準備としてSTをHighへ戻す.
この構成により,高周波領域においても各サンプルの時刻はTRG(ハードウェアイベント)により規定され,
CPU介在はフレーム境界(開始/完了)に限定される.

\subsection{TIM--ADC--DMAの連携機構}
\label{subsec:tim_adc_dma_link}
図\ref{fig:c12880ma_timing}に示すように,C12880MAはCLKに同期してTRGを生成する.
本研究では,汎用タイマ(TIM2)のPWM出力によりCLKを生成し,
別タイマ(TIM15)でTRGエッジを取り込み,そのイベントをADCの外部トリガとして用いる.
ADCは外部トリガ入力により変換を開始し,変換結果はDMAによりメモリへ逐次転送されるため,
TRG周期に同期した一様なサンプリングがCPUの介在なしに実行される.

重要なのは,ADC変換の開始条件が「ソフトウェア呼出」ではなく「TRGエッジ」により決まる点である.
これにより,ISR内処理やタスクスケジューリングに起因する遅延・ばらつきが
サンプリング位相へ直接投影されることを抑制し,
\SI{5}{\mega\hertz}級の連続取得においてもタイミング決定性を維持できる.

\subsection{ST立下り後の開始遅延と有効画素ウィンドウ補正}
\label{subsec:effective_pixel_window}
C12880MAのデータシートでは,ST立下り後の所定回数(89回)のTRGを経てVIDEOが有効化される.
一方,本実装ではST立下り後にADC/DMAを開始するため,
ST遷移からDMA開始までのソフトウェア起動遅延が,画素列上の開始位置に固定オフセットとして現れる.

本研究環境では,このオフセットはTRG周期換算で約4画素分に相当することが実測により一貫して確認された.
そのため,DMAバッファ上で有効画素が安定して出現する位置は理論値(89番目)ではなく,
およそ85番目付近となる.本研究ではこの挙動を系統的な固定遅延として扱い,
先頭85画素を無効画素として除外することで,有効スペクトルデータを安定に抽出した.

\subsection{割り込み負荷を伴うシステムへの拡張(論理ゲートによる完全同期)}
\label{subsec:and_gate_future}
本章の評価では,高速取得中に追加の高負荷割り込み処理(通信スタックや制御周期割り込み等)を介在させず,
時間決定性を優先した構成としている.この前提下では,前節の画素オフセットはほぼ一定であり,
有効画素ウィンドウ補正として安定に吸収できる.

一方,将来的に割り込み負荷が増大し,ST遷移からADC/DMA開始までのレイテンシが変動する場合,
画素シフトが非定常となり,定数補正のみでは不十分となる可能性がある.
その場合には,ST信号と「ADC/DMAのARM完了(開始準備完了)」状態を論理回路(例:与ゲート)で合成し,
「ST=LowかつARM完了」の条件成立後にのみTRG(もしくはADCトリガ経路)を有効化する
ハードウェアゲーティングにより,フレーム開始位置を完全に決定論的に固定できる.
本研究は現段階でこの拡張を要さないが,システム統合要件の変化に応じて導入可能である.


\section{実装要点}
\label{sec:implementation_points}
本節では,提案アーキテクチャをMCU上で安定動作させるための実装要点を述べる.

\subsection{ハードウェアトリガ連鎖(Timer--ADC--DMA)}
\label{subsec:hw_trigger_chain}
ADCは外部トリガ入力により変換を開始し,変換結果をDMAでメモリへ連続転送する.
本システムでは,TIM2のPWMによりC12880MAへCLKを供給し,
TRGはTIM15でエッジ同期イベントとして取得した上で,ADC外部トリガへ接続する.
このとき,ADCの変換開始はTRGエッジにより駆動されるため,
CPUが周期ごとにADCを起動する必要がなく,\SI{5}{\mega\hertz}級の高頻度でもサンプリング位相の決定性を維持できる.

また,DMAは各変換結果を逐次バッファへ転送し,フレーム終端はDMA転送完了で検知する.
これにより,CPU介在はフレーム境界(開始・完了)に限定され,
周期イベントごとのISR負荷が性能制約となる問題を本質的に回避できる.

\subsection{STM32H7系列におけるキャッシュ・コヒーレンシ対策}
\label{subsec:cache_coherency}
STM32H7系列ではデータキャッシュによりDMA転送データの可視性が損なわれる場合がある.
本研究ではDMA転送先バッファをDTCM(Data Tightly Coupled Memory)領域へ配置し,
キャッシュの影響を受けないメモリによりデータ整合性を確保した.
併せて,必要に応じてキャッシュ無効化あるいはキャッシュクリーン/インバリデートを実施し,
連続取得時の欠落・破損を抑制した.

\section{性能検証}
\label{sec:performance_eval}
本節では,提案アーキテクチャが高速取得を実現していることを,
(i) 信号タイミングの実測,(ii) 連続取得時のデータ完全性,(iii) スペクトルの再現性,
の3観点から検証する.
なお,本実験に用いたSTM32 NUCLEO-H723ZG開発ボードとセンサユニットの
接続インタフェース詳細については付録を参照する.


\subsection{STM32F446REでのアーキテクチャ検証}
従来方式では\SI{130}{\kilo\hertz}が限界であったSTM32F446REに提案アーキテクチャを実装したところ,
\SI{0.5}{\mega\hertz}および\SI{1}{\mega\hertz}での安定取得を確認した.
これにより,従来方式の主要制約が割込み起動遅延に起因していたこと,
および提案アーキテクチャがその解消に有効であることが示された.
なお,本MCUに搭載されるADCの仕様により,達成可能な最大周波数は理論上\SI{1.5}{\mega\hertz}程度に制限される
(データシートに基づく上限).

\subsection{STM32H723ZGでの\protect\SI{5}{\mega\hertz}高速取得}
\label{subsec:eval_h7}
最大\SI{5}{\mega\hertz}級のサンプリングを実現するため,STM32H723ZGに同アーキテクチャを実装した.
前述のキャッシュ・コヒーレンシ対策としてDMAバッファをDTCMへ配置し,
ADCのハードウェアキャリブレーションおよびトリガ設定を最適化した.
その結果,\SI{5}{\mega\hertz}での連続取得に成功した.

\subsection{オシロスコープのスクリーンショットによる\protect\SI{5}{\mega\hertz}動作確認}
\label{subsec:timing_scope_screenshot}

Fig.~\ref{fig:scope_clk_trg_video_eos}に,\SI{5}{\mega\hertz}動作時のオシロスコープ画面を示す.
CH1をTRG信号,CH2をCLK信号,CH4をVIDEO信号,CH3をEOS信号とする.
なおスクリーンショットではTRGおよびCLKの振幅が実際より小さく表示されているが,
これはプローブ減衰設定とスケール設定の不整合に起因する表示上の問題であり,
本図は主としてタイミング関係(周期・位相・エッジ同期)の確認に用いる.
強光照射時と遮光時を比較すると,EOS近傍でVIDEO出力が停止する挙動が確認できる.

\begin{figure}[t]
  \centering
  \begin{subfigure}[b]{0.49\linewidth}
    \centering
    \includegraphics[width=\linewidth]{figures/4/SCRN0038.PNG}
    \caption{Strong illumination}
    \label{fig:eos_strong}
  \end{subfigure}
  \hfill
  \begin{subfigure}[b]{0.49\linewidth}
    \centering
    \includegraphics[width=\linewidth]{figures/4/SCRN0037.PNG}
    \caption{Dark condition}
    \label{fig:eos_dark}
  \end{subfigure}
  
  \caption{Oscilloscope screenshots of the End-of-Scan (EOS) timing at \SI{5}{\mega\hertz}. 
  (a) Under strong illumination, VIDEO (green) drops before EOS (blue) rises. 
  (b) In dark condition. 
  Note: Yellow: TRG, Magenta: CLK. The displayed amplitudes of CLK/TRG are compressed due to probe scaling, but the timing relationship is preserved.}
  \label{fig:scope_clk_trg_video_eos}
\end{figure}

\subsection{CSVファイルから再描画したEOS信号,ST信号およびVIDEO信号の波形}
\label{subsec:timing_csv_plots}
オシロスコープから保存したCSVデータを用い,制御信号(EOS, ST)と出力信号(VIDEO)の関係を再描画した.

まず,Fig.~\ref{fig:eos_video_high_low}にEOSとVIDEOの同時波形を示す.
強光照射時((a))と遮光時((b))のいずれにおいても,
EOS信号の立ち上がりエッジに同期してVIDEO出力が停止(ベースラインへ復帰)しており,
ゲートロジックによる読出停止が機能していることを確認した.

次に,様々な照明環境および光源距離におけるST信号とVIDEO信号の応答特性をFig.~\ref{fig:st_video_combined_conditions}に示す.
Fig.~\ref{fig:st_video_combined_conditions}(a)の遮光条件ではベースラインノイズのみが観測され,
同(b)の自然光下では光源のスペクトルピークが明瞭に捉えられている.
また,光源距離を変化させた同(c)(\SI{5}{cm},強光相当)および同(d)(\SI{25}{cm},弱光相当)を比較すると,
光量に応じてVIDEO信号の振幅が適切に変化しており,
本システムが広いダイナミックレンジで線形応答性を維持していることが確認された.

\begin{figure}[h]
  \centering
  \begin{subfigure}[b]{0.49\linewidth}
    \centering
    \includegraphics[width=\linewidth]{figures/4/eos_video_high.png}
    \caption{Strong illumination}
    \label{fig:eos_high}
  \end{subfigure}
  \hfill
  \begin{subfigure}[b]{0.49\linewidth}
    \centering
    \includegraphics[width=\linewidth]{figures/4/eos_video_low.png}
    \caption{Dark condition}
    \label{fig:eos_low}
  \end{subfigure}
  
  \caption{Replotted waveforms from CSV (EOS and VIDEO) at \SI{5}{\mega\hertz}. 
  (a) Under strong illumination. 
  (b) Under dark condition. 
  In both cases, the VIDEO output (red) becomes inactive precisely at the rising edge of the EOS event (green), verifying the readout termination logic.}
  \label{fig:eos_video_high_low}
\end{figure}

\begin{figure}[t]
  \centering
  % Row 1: Baseline and Natural Light
  \begin{subfigure}[b]{0.48\linewidth}
    \centering
    \includegraphics[width=\linewidth]{figures/4/st_video_black.png}
    \caption{Dark condition}
    \label{fig:cond_dark}
  \end{subfigure}
  \hfill
  \begin{subfigure}[b]{0.48\linewidth}
    \centering
    \includegraphics[width=\linewidth]{figures/4/st_video_nature_light.png}
    \caption{Natural light}
    \label{fig:cond_nature}
  \end{subfigure}
  
  \vspace{1em} % Add some vertical space between rows

  % Row 2: Distance/Intensity comparison
  \begin{subfigure}[b]{0.48\linewidth}
    \centering
    \includegraphics[width=\linewidth]{figures/4/st_video_5cm_light.png}
    \caption{Strong illumination (\SI{5}{cm})}
    \label{fig:cond_5cm}
  \end{subfigure}
  \hfill
  \begin{subfigure}[b]{0.48\linewidth}
    \centering
    \includegraphics[width=\linewidth]{figures/4/st_video_25cm_light.png}
    \caption{Low illumination (\SI{25}{cm})}
    \label{fig:cond_25cm}
  \end{subfigure}
  
  \caption{ST and VIDEO signal waveforms under different lighting and distance conditions. 
  (a) Dark condition showing baseline noise. 
  (b) Natural light showing spectral peaks. 
  (c) High intensity at \SI{5}{\cm} distance (equivalent to strong illumination). 
  (d) Low intensity at \SI{25}{\cm} distance. 
  The results confirm the system's ability to capture spectral features and its dynamic response to varying light intensities.}
  \label{fig:st_video_combined_conditions}
\end{figure}

\subsection{サンプリングレート成立性の検討(ADC仕様に基づく)}
\label{subsec:adc_feasibility}
センサに供給するクロックが\SI{5}{\mega\hertz}であるため,更新周期は
\[
T_{\mathrm{period}}=\frac{1}{\SI{5}{\mega\hertz}}=\SI{200}{\nano\second}
\]
となる.データシートによればVIDEO信号が安定しているのはTRG立上りエッジを中心とした半周期程度であるため,
サンプリング可能時間幅は概ね\(\SI{100}{\nano\second}\)となる.
本研究ではサンプリング時間\(\SI{2.5}{}\)サイクル,変換時間\(\SI{12.5}{}\)サイクル(12-bit)とし合計15サイクルを要するため,
必要なADCクロック周波数\(f_{\mathrm{ADCK}}\)は
\begin{equation}
  f_{\mathrm{ADCK}} > \frac{15}{\SI{200}{\nano\second}} = \SI{75}{\mega\hertz}
\end{equation}
となる.STM32H723ZGのデータシート\cite{ref:stm32h723zg}において12-bit ADCの最大クロック周波数が\SI{75}{\mega\hertz}と規定されていることから,
設定が理論上成立する.実測により本条件で安定動作することを確認した.


\begin{table}[!tbp]
  \centering
  \caption{Acquisition frequency of the proposed architecture}
  \label{tab:acq_freq_proposed}
  \setlength{\tabcolsep}{5pt}
  \renewcommand{\arraystretch}{1.15}
  \begin{tabular}{l l l}
    \toprule
    Method & MCU & Achieved frequency \\
    \midrule
    Proposed architecture & STM32F446RE & \SI{1.5}{\mega\hertz} (theory) \\
    Proposed architecture & STM32H723ZG & \SI{5.0}{\mega\hertz} (achieved) \\
    \bottomrule
  \end{tabular}
\end{table}


\section{その他センサのデータ取得と統合}
\label{sec:other_sensors}
分光データと同時に,生育環境の基礎量として温度・湿度・気圧および\(\mathrm{CO_2}\)濃度を取得するため,
BME280およびS300L-3Vを併用した.これら環境センサの取得処理そのものは先行研究で確立された実装に基づくが,
本研究ではセンサユニットのMCUをSTM32系列へ変更したことに伴い,
デバイスドライバおよび周辺制御(I\(^2\)C設定,割込み・DMA設定,タイマによる周期実行)を再設計し,
既存処理をSTM32環境へポーティングした.具体的には,S300L-3VやBME280はI\(^2\)C通信により所定周期で計測する.
各計測値にはMCU内の単調増加タイムスタンプを付与し,
分光フレームと同一の時刻系でログ化することで,走行中のマルチモーダル計測データとして統合可能とした.
% 収集したデータは上位計算機へ周期的に送信し記録する.通信方式およびデータフォーマットの詳細は,
% 再現性確保の観点から付録にまとめる.

\section{まとめ}
\label{sec:sensor_unit_summary}
本章では,C12880MAを用いた高速分光センシングの実現に向け,
割込み駆動方式の制約を再現的に示した上で,
連続CLK下でのST制御とTRG同期ハードウェアトリガに基づくADC/DMAストリーミング取得系を構築した.
さらに,ST立下り後のADC/DMA開始に伴う画素レベルの開始オフセットを実測により定量化し,
有効画素ウィンドウ補正として取り込むことで,高速取得時のスペクトル整合性を確保した.
STM32H723ZGに実装し,\SI{5}{\mega\hertz}での安定取得をオシロスコープおよびCSV再描画により検証した.
また,将来的に割り込み負荷が増大して遅延が非定常化する場合には,
論理ゲートによるハードウェアゲーティングへ拡張することでフレーム開始位置を決定論的に固定できることを示した.
次章では,本センサユニットにより取得したデータを用いた環境マッピングおよび統合評価について述べる.

\end{document}