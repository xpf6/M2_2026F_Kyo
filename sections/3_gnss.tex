%! TEX root = ../main.tex
\documentclass[main]{subfiles}

\begin{document}

\chapter{GNSS品質監視に基づくロバストな自己位置推定}
\label{ch:method}
本章では,農業用ハウス環境のような半構造環境において,
RTK-GNSSの測位品質が遮蔽・マルチパス・差分補正途絶等により間欠的に変動する課題に対処するため,
LiDAR--Inertial Odometry(LIO)とRTK-GNSSを統合した自己位置推定系を構築する.
特に,受信機状態量に基づく可用性判定と,LIOとの短時間増分整合性に
基づく完全性監視(Integrity Monitoring)を組み合わせ,見かけ上は高品質(小共分散)でありながら
実際には大きく偏る誤測位(HMI: Hazardously Misleading Information)を抑制する
GNSS品質監視モジュール(GNSS Supervisor)を提案する.
提案手法は,バックエンドのグラフ最適化に投入するGNSS制約をオンラインに選別・重み付け調整し,
長時間走行における自己位置推定の安定性向上を目的とする.

\section{システム構成}
\label{sec:system_overview}
本研究で構築した移動ロボットシステムは,
農業用ハウスを含む半構造化環境において,
自己位置推定と生育環境センシングを同時に実現することを
目的として設計されている.
本システムは,周囲環境およびロボット自身の運動状態を
観測するハードウェア系と,
それらの観測情報を統合して位置推定およびマッピングを
行うソフトウェア系から構成される.

農業環境では,GNSS 測位品質が遮蔽やマルチパスの影響により時間的に大きく変動し,
またハウス内通路のような単調な幾何構造に起因して,
LiDAR を用いた自己位置推定が幾何学的に縮退する局面が生じ得る.
このような環境特性を踏まえ,本研究では,センサ構成そのものの冗長化ではなく,
観測品質の変動を前提とした情報統合手法を採用する.

以下では,まず移動ロボットプラットフォームおよび搭載センサについて述べ,
次に LiDAR--IMU オドメトリ,ナビゲーションシステム,
および NSS品質監視に基づくソフトウェアアーキテクチャについて説明する.

\subsection{移動ロボットプラットフォームと搭載センサ}
\label{subsec:hardware}
移動ロボットには,不整地走行を想定したAGVシャーシを用い,
圃場内を巡回しながら各種センサデータを取得できる構成とした.
AGVシャシではGmade 社のGS02を採用した.
環境認識・自己位置推定のために全方位ソリッドステート式3D-LiDAR(Livox社Mid-360)を使用する.
LiDAR 点群は地図生成および自己位置推定の主要な観測情報として用いる.
LiDAR に内蔵された IMU は,
点群の運動歪み補正および短時間の姿勢推定に利用される.

絶対位置計測には2周波RTK-GNSS受信機(u-blox社ZED-F9P)を用い,
基準局から配信されるRTCM補正情報を利用することで,
屋外環境においてセンチメータ級の測位精度を目指す.
また,RTCM補正情報の取得にあたり,
森林総合研究所が公開しているRTK基準局サービスを利用した\cite{jforest_rtk}.
RTCM補正情報は無線LANを介してロボットに入力され,
GNSS受信機によりリアルタイムで適用される.

一方で,農業用ハウス周辺では,
作物や構造物による遮蔽およびマルチパスの影響により,
測位状態や共分散が時間的に大きく変動することが確認されている.
そのため,本研究では GNSS 観測を常に信頼可能な情報として扱うのではなく,
後述する品質監視に基づいて,
自己位置推定への寄与を動的に制御する.

また,ロボットの駆動にはBLDC(Brushless Direct Current)モータを使用するが,
本研究では不整地でのスリップが顕著であったため,
車輪オドメトリは自己位置推定の主要情報源として用いない.
センサデータ取得と推定処理はRaspberry Pi 4B上で実行し,無線LANを介して通信を行う.

\subsection{ソフトウェアアーキテクチャ}
\label{subsec:software}
本システムはROS~2(Robot Operating System 2)\cite{ref:ROS2} を基盤とし,自己位置推定・地図生成・自律走行をモジュール化して構成する.
自己位置推定およびマッピングは,フロントエンドとしてLiDAR--IMUオドメトリを
用いて高頻度の相対運動推定を行い,
バックエンドとしてRTAB-Map\cite{ref:rtabmap}によりループ閉合と因子グラフ最適化を実行する.
さらに,GNSS品質監視モジュール(GNSS Supervisor)により観測の採否および共分散調整を行ったRTK-GNSS観測を,
バックエンド最適化の位置制約として導入する.
自律走行はNavigation 2\cite{ref:Navigation2}フレームワークを用いて実装し,
地図およびロボット姿勢に基づく経路計画と追従制御を行う.

\subsubsection{ROS~2}
\label{subsubsec:ros2}
ソフトウェア基盤にはROS~2を採用する.
ROS 2は,ロボットアプリケーション構築のためのソフトウェアラ
イブラリとツールのセットである.
ROS 2 は ROS 1 に続く新しいバージョンで,データ配
信サービス(DDS)に基づき,セキュリティ,リアルタイムサポート,マルチロボット通信,
非理想的なネットワーク環境での操作など,信頼性の高いロボティクスシステム構築に不可
欠な機能を実現することができる.

\subsubsection{Navigation2}
\label{subsubsec:navigation}
自律走行にはNavigation2を用いる.
Navigation2は,グローバル経路計画,ローカルプランニング,
Behavior Treeによるタスク実行を統合的に提供する.
本システムでは,バックエンドSLAMが提供する地図座標系における自己位置に基づいて経路計画を行い,
LiDAR観測に基づく局所障害物回避を併用する.
局所コストマップ生成にはSpatio-Temporal Voxel Layer(STVL)\cite{ref:stvl}を用い,
3次元点群を時系列に保持するボクセル表現により,近傍障害物の占有推定を行う.
これにより,農業環境における不整地や局所的な遮蔽物に対しても,
走行時の安全性と追従安定性の向上を図る.

\subsubsection{LiDAR--IMUオドメトリおよびSLAM}
\label{subsubsec:lio_slam}
フロントエンドにはFAST-LIO2\cite{ref:fast_lio2}を採用し,
LiDAR点群とIMU計測から高頻度の相対オドメトリを推定する.
FAST-LIO2は,高速なスキャンマッチングとIMU統合により,
リアルタイム動作に適したオドメトリを提供する.

バックエンドにはRTAB-Mapを用い,
ループ閉合検出と因子グラフ最適化により地図整合性を向上させる.
RTAB-Mapを採用した理由は,(i)ループ閉合に基づく長期ドリフト抑制を実装容易に実現できること,
(ii)複数センサ入力(LiDAR,GNSS等)を制約として扱える枠組みを有すること,
(iii)将来的に視覚センサ等を追加した場合でも入力モダリティの拡張が比較的容易であり,
農業環境マッピングへの発展性が高いこと,の3点である.

\subsubsection{座標系とTF構成}
\label{subsubsec:tf}
本システムの座標系は以下のように構成する.
\begin{itemize}
  \item FAST-LIO2:\texttt{odom} $\rightarrow$ \texttt{base\_link}(高頻度相対オドメトリ)
  \item RTAB-Map:\texttt{map} $\rightarrow$ \texttt{odom}(バックエンド最適化により更新される変換)
  \item 静的外部パラメータ:\texttt{base\_link} $\rightarrow$ \texttt{lidar},\texttt{base\_link} $\rightarrow$ \texttt{gps\_antenna}
\end{itemize}
これにより,制御・追従は\texttt{odom}系の連続性を保持しつつ,
地図参照のナビゲーションおよび環境マッピングは\texttt{map}系でのグローバル整合を利用できる.

\subsubsection{データフロー}
\label{subsubsec:dataflow}
主なデータフローを以下に示す.
\begin{itemize}
  \item FAST-LIO2:点群・IMU $\rightarrow$ $\,^{\texttt{odom}}\!T_{\texttt{base}}(t)$
  \item GNSS Supervisor:\texttt{NavSatFix}+受信機状態量+LIOオドメトリ $\rightarrow$ 採否・共分散調整済みGNSS観測
  \item RTAB-Map:点群+LIOオドメトリ+品質監視後GNSS $\rightarrow$ ループ閉合/因子グラフ最適化 $\rightarrow$ \texttt{map}$\rightarrow$\texttt{odom},最適化軌跡
  \item Nav2:地図(\texttt{map})+自己位置(\texttt{map}系)+局所コストマップ(STVL) $\rightarrow$ 経路計画/追従制御
  \item 生育環境マッピング:最適化軌跡に基づき環境センサ値を\texttt{map}座標系へ投影
\end{itemize}


\section{農業環境における課題と故障モード}
\label{sec:problem_definition}
農業用ハウス環境におけるRTK-GNSSは,多くの時間帯で高精度に利用可能である一方,
遮蔽・マルチパス・差分補正の途絶等により,測位品質が間欠的に劣化する.
本研究では,この品質劣化を単一の現象として扱わず,
オンラインで区別可能な故障モードとして次の3種類に整理する.

\begin{enumerate}
  \item \textbf{GNSS利用不可能}\\
  no-fix,使用衛星数不足,差分補正の途絶(RTCM ageの増大)等により,
  測位が成立しない,あるいは測定として利用できない状態である.これは測定欠落に相当する.
  \item \textbf{危険な誤誘導情報(HMI: Hazardously Misleading Information)/外れ値混入}\\
  受信機はRTK-FIX等の状態を報告し,共分散も小さいにもかかわらず,実際の測位が大きく偏り,
  推定軌跡に不連続な位置の飛び(位置ジャンプ)を生じる状態である.本研究の焦点は,
  この見かけ上は良好だが誤っている測位の抑制にある.
  \item \textbf{LIOの幾何学的退化(LiDARデジェネラシー)と不整地擾乱}\\
  ハウス内通路等では点群幾何が単調となり,可観測性が低下してLIOが退化し得る.
  また不整地では車輪スリップが顕著であり,実験では,Wheel Odometryは自己位置推定源としては不適切と判断した.
\end{enumerate}

本研究の目的は,GNSSが高精度だが,利用不可が間欠的に発生し,まれにHMIが出現する統計的特性をもつ状況下で,
後端のグラフ最適化を汚染しないことを最優先としつつ,
高品質RTK拘束によりLIOの長期ドリフトを抑制する統合推定系を構築することである.
そのために,後端へ注入するGNSS拘束をオンラインで選別する品質監視モジュールを設計する.


\section{提案手法:GNSS Supervisorの設計}
\label{sec:supervisor}
Supervisorは,可用性判別,完全性監視,LIO信頼度から構成される.可用性ゲートは利用不可を遮断し,
完全性ゲートはHMI/外れ値混入を抑制する.ただしLIOも退化し得るため,
LIO信頼度に基づき完全性ゲートの厳しさを調整し,
必要に応じて完全性ゲートを停止して可用性判別のみへ切り替える.

\subsection{可用性ゲート}
\label{subsec:availability_gate}
可用性ゲートは,LIO等の外部推定に依存せず,
受信機自身が提供する状態量に基づく入力有効性検査として実装する.
目的は,測位不可や補正途絶に起因する利用不可を確実に遮断し,
後端グラフへ不正な拘束が注入されることを防ぐことである.

\subsubsection{状態量}
受信機から次の状態量を取得・記録し,判別に用いる.
\begin{itemize}
  \item 測位状態(no-fix,3D-fix,RTK-float,RTK-fix等)
  \item 使用衛星数(あるいは測位に寄与する衛星数)
  \item 搬送波解の状態
  \item 差分補正の受信状態(RTCM age,途絶時間)
  \item 位置推定共分散(NavSatFixに付随するposition\_covariance)
\end{itemize}
NavSatFixのみで不足する場合は,
受信機拡張ステータス(UBX等)を併記してログ化する.

\subsubsection{4状態モデルとヒステリシス}
GNSS入力の状態をS0--S3に分類する.
\begin{itemize}
  \item \textbf{S0:NO\_MEAS(利用不可)}\\
  no-fix,衛星数不足,RTCM age超過,共分散が未定義(NaN)等.
  \item \textbf{S1:DEGRADED(降級可用)}\\
  RTK-float/3D-fix等で測位は成立するが精度が劣化している状態.
  \item \textbf{S2:CANDIDATE(回復直後の観察状態)}\\
  S0から回復した直後であり,安定性を確認するための遷移状態.
  \item \textbf{S3:TRUSTED(可用)}\\
  RTK-fixが安定し,状態量が良好である状態(完全性ゲート通過を前提に最終採用).
\end{itemize}

判定の過敏さを避けるため,ヒステリシスを導入する.例えば,
\begin{itemize}
  \item 利用不可条件が連続$N_{\mathrm{bad}}$回成立 $\Rightarrow$ S0へ遷移し,GNSSを封鎖
  \item 利用可能条件が連続$M_{\mathrm{good}}$回成立 $\Rightarrow$ S2へ遷移し,観察後にS3候補
\end{itemize}
とする.

\subsection{完全性ゲート(Integrity Gate):短時間増分整合性検定}
\label{subsec:integrity_gate}
完全性ゲートは,受信機状態が良好に見えるにもかかわらず外れ値を含むHMIを抑制するための機構である.本研究では,絶対座標の一致を仮定せず,\textbf{短時間窓の位置増分}を比較することで外れ値混入を検出する.増分形式を採用することで,初期方位や座標原点の差異に対して頑健となる.

\subsubsection{GNSSの局所座標化}
完全性検定は局所直交座標系(ENU)で実施する.$t$時刻のGNSS測位を$\,^{\texttt{enu}}\bm{p}_{\mathrm{gnss}}(t)$とし,基準点(原点)は可用状態S3が成立した初回の測位(あるいは観察状態S2を経て安定と判断した測位)を用いて設定する.局所座標化の実装は\ref{sec:enu_conversion}節に述べる.

\subsubsection{増分残差とNIS}
時間窓$\Delta T$に対して,LIOおよびGNSSの位置増分を次式で定義する.
\begin{align}
  \Delta \bm{p}_{\mathrm{lio}}(t)  &= \bm{p}_{\mathrm{lio}}(t)-\bm{p}_{\mathrm{lio}}(t-\Delta T), \\
  \Delta \bm{p}_{\mathrm{gnss}}(t) &= \bm{p}_{\mathrm{gnss}}(t)-\bm{p}_{\mathrm{gnss}}(t-\Delta T).
\end{align}
ここで$\bm{p}_{\mathrm{gnss}}(t)$はENU座標のGNSS位置,$\bm{p}_{\mathrm{lio}}(t)$はLIOオドメトリを同一座標系で表現した位置(\texttt{odom}系)である.増分残差を
\begin{equation}
  \bm{r}(t)=\Delta \bm{p}_{\mathrm{gnss}}(t)-\Delta \bm{p}_{\mathrm{lio}}(t)
\end{equation}
とし,共分散を
\begin{equation}
  \bm{S}(t)=\Sigma_{\mathrm{gnss}}(t)+\Sigma_{\mathrm{lio}}(t)
\end{equation}
で近似する.正規化イノベーション二乗(NIS)を
\begin{equation}
  d(t)=\bm{r}(t)^{\mathsf{T}}\bm{S}(t)^{-1}\bm{r}(t)
\end{equation}
として計算し,$d(t)>\gamma$のとき外れ値疑いと判定する.$\gamma$は自由度3の$\chi^2$分布に基づいて設定し,$\Delta T$は0.5--1.5\,sの範囲で実測により調整する(本研究では初期値として$\Delta T=1.0$\,sを用いる).

\subsubsection{down-weight優先の抑制とヒステリシス}
本システムでは,GNSSが大部分の時間帯で有効であるため,誤報(False Alarm)による過剰封鎖は絶対拘束の喪失を招く.そこで,判定結果の反映は\textbf{down-weight(寄与抑制)を優先}し,明確な外れ値と判断した場合のみdropする.

具体的には,外れ値疑いの程度に応じてスケール$\alpha(t)\ge 1$を設定し,
\begin{equation}
  \Sigma_{\mathrm{gnss}}'(t)=\alpha(t)\,\Sigma_{\mathrm{gnss}}(t)
\end{equation}
として後端へ入力する.$\alpha(t)$は段階設定(例:$1,\,10,\,100$)とし,再現性を確保する.またヒステリシスとして,
\begin{itemize}
  \item 不合格が連続$N_{\mathrm{bad}}$回 $\Rightarrow$ 封鎖(dropもしくは強いdown-weight)
  \item 合格が連続$M_{\mathrm{good}}$回 $\Rightarrow$ 解封(通常入力へ復帰)
\end{itemize}
を適用する.

\subsection{LIO信頼度(LIO Confidence)による退化保護}
\label{subsec:lio_confidence}
長廊等でLIOが退化している局面では,LIOを参照として完全性検定を行うと誤判定が増加し得る.そこでLIO信頼度$C_{\mathrm{lio}}\in[0,1]$を導入し,完全性ゲートの厳しさを調整する.

\subsubsection{信頼度指標(オンラインproxy)}
実装容易性と再現性を重視し,$C_{\mathrm{lio}}$は次のいずれか(または組合せ)で定義する.
\begin{itemize}
  \item \textbf{内部統計に基づく指標(取得可能な場合)}:有効特徴点数,残差統計,反復回数,推定共分散の増大など.
  \item \textbf{点群幾何に基づく退化proxy(内部量が取得困難な場合)}:最近傍$K$フレーム点群に対してPCAを行い,固有値$\lambda_1\ge\lambda_2\ge\lambda_3$の比(例:$\rho=\lambda_3/\lambda_1$)を退化指標とする.$\rho$が小さい場合,強い一次元性を示し長廊退化を疑う.
\end{itemize}

\subsubsection{信頼度に基づく調整規則}
$C_{\mathrm{lio}}$が低い場合は,次のいずれかで完全性ゲートを緩和する.
\begin{enumerate}
  \item $\Sigma_{\mathrm{lio}}$の拡大:$C_{\mathrm{lio}}$低下に応じて$\Sigma_{\mathrm{lio}}$を拡大し,NISが過敏に反応することを防ぐ.
  \item 閾値$\gamma$の緩和:$C_{\mathrm{lio}}$低下に応じて$\gamma$を増大させる.
  \item 完全性ゲートの一時停止:$C_{\mathrm{lio}}<\tau_C$のとき,完全性ゲートを停止し,可用性ゲートのみによる入力管理へ切り替える.
\end{enumerate}
これにより,LIO退化局面で「誤った参照に基づいてGNSSを恒常的に否定する」状況を回避する.

\section{WGS84から局所ENUへの変換}
\label{sec:enu_conversion}
Supervisor内部の整合性検定では,GNSSを局所直交座標へ変換する必要がある.出力はRTAB-Mapへ\texttt{NavSatFix}として与えるため,\textbf{変換はSupervisor内部のみに閉じる}.

\subsection{GeographicLibによるWGS84$\rightarrow$ENU(推奨)}
可用状態S3が初めて成立した時刻の測位$(\varphi_0,\lambda_0,h_0)$を原点として設定し,各時刻の測位$(\varphi,\lambda,h)$をENUへ変換して$\,^{\texttt{enu}}\bm{p}_{\mathrm{gnss}}(t)$を得る.この方式は構成が単純であり,姿勢入力を必要としないため,実装・検証が容易である.

\subsection{navsat\_transform\_nodeの利用(代替)}
ROS標準ノードを用いて\texttt{NavSatFix}をローカル座標へ変換する方法もある.ただし方位入力(IMUあるいはLIO由来yaw)と座標系整合の要件が増えるため,本研究では原則としてGeographicLib方式を採用する.

\section{RTAB-MapへのGNSS拘束注入と三値出力(Pass/Inflate/Drop)}
\label{sec:rtabmap_injection}
GNSSは後端グラフ最適化における\textbf{疎な全局拘束}として扱う.Supervisorの出力は,RTAB-Mapへ入力するGNSS拘束の取り扱いを三値で決定する.
\begin{itemize}
  \item \textbf{Pass}:\texttt{NavSatFix}をそのまま出力し,RTAB-Mapが通常重みで拘束として利用する.
  \item \textbf{Inflate}:\texttt{position\_covariance}を$\alpha(t)$倍に拡大して出力し,拘束の重みを弱める(ソフト抑制).
  \item \textbf{Drop}:当該時刻のGNSSを出力しない(またはRTAB-Mapが無視できるフラグを付与する)ことで拘束注入を行わない(ハード抑制).
\end{itemize}
これにより,利用不可・外れ値混入のGNSSが後端地図を汚染することを防止しつつ,大部分の高品質RTK拘束を最大限活用できる.


\section{実験と評価}
実験フィールド(信州大学農場ハウス)、使用したAGV、データ収集シナリオ
(GNSS良好時、不良時、ハウス内外移動時)を詳細に記述する。

評価に用いたGround Truthの定義(例:GNSS良好時の高精度軌跡、あるいは外部計測機器)を明確にする。
\subsection{実験条件}
\subsection{比較手法}
比較手法: (1) LIO (+Wheel Odom) のみ、(2) LIO+Wheel+GNSSのナイーブな融合(品質無視)、(3) 提案手法(品質監視付き融合)の3つを用意する。

各シナリオのデータセットに対して3手法を適用し、得られた軌跡をGround Truthと比較する。

結果の提示: (1) 軌跡比較図(Fig. 
%\ref{fig:trajectory_comparison}
)、
(2) 定量評価指標(RMSE, 最大誤差など)の比較表(Table 
%\ref{tab:localization_rmse}
)を示す。
\subsection{結果と考察}
実験結果に基づき、
提案手法(3)が比較手法(1)(2)に対して優位性を持つことを明確に論証する。
特に、GNSS品質が悪化した際に、ナイーブ融合(2)が破綻するのに対し、提案手法(3)が安定して精度を維持できることを強調する。

GNSS品質監視モジュールの閾値設定などのパラメータの妥当性についても議論する

\end{document}