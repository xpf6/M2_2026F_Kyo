%! TEX root = ../main.tex
\documentclass[main]{subfiles}

\begin{document}

\chapter{GNSS品質監視に基づくロバストな自己位置推定}
\label{ch:method}
本章では,農業用ハウス環境のような半構造環境において,
RTK-GNSSの測位品質が遮蔽・マルチパス・差分補正途絶等により間欠的に変動する課題に対処するため,
LiDAR--Inertial Odometry(LIO)とRTK-GNSSを統合した自己位置推定系を構築する.
特に,受信機状態量に基づく可用性判定と,LIOとの短時間増分整合性に
基づく完全性監視(Integrity Monitoring)の考え方を援用し,
見かけ上は高品質(小共分散,RTK-FIX相当の状態)であっても,
遮蔽・マルチパスに起因するNLOS(Non-Line-of-Sight)環境下で大きく偏り得る誤測位(大偏差外れ値)を抑制する
GNSS品質監視モジュール(GNSS Supervisor)を提案する.
提案手法は,バックエンドのグラフ最適化に投入するGNSS制約をオンラインに選別・重み付け調整し,
長時間走行における自己位置推定の安定性向上を目的とする

\section{システム構成}
\label{sec:system_overview}
本研究で構築した移動ロボットシステムは,
農業用ハウスを含む半構造化環境において,
自己位置推定と生育環境センシングを同時に実現することを
目的として設計されている.
本システムは,周囲環境およびロボット自身の運動状態を
観測するハードウェア系と,
それらの観測情報を統合して位置推定およびマッピングを
行うソフトウェア系から構成される.
全体構成と主要な情報流れをFig.~\ref{fig:system_overview}に示す.
以降ではハードウェア,ソフトウェア,各座標系の関係を順に述べる.

\begin{figure}[t]
  \centering
  \includegraphics[width=0.9\linewidth]{figures/3/system.png}
  \caption{Overview of the system architecture and data flow. The PC-side ROS~2 stack runs RTAB-Map (SLAM) and Navigation2 with STVL (planning and control), while the AGV-side ROS~2 stack runs FAST-LIO2 to provide registered point clouds and odometry. Sensor topics (LiDAR/IMU/GNSS/environment) are exchanged via ROS~2, and velocity commands are sent as \texttt{cmd\_vel} (\texttt{geometry\_msgs/Twist}). The TF/\texttt{map} frame relationship used for navigation is also illustrated.}
  \label{fig:system_overview}
\end{figure}


農業環境では,GNSS 測位品質が遮蔽やマルチパスの影響により時間的に大きく変動し,
またハウス内通路のような単調な幾何構造に起因して,
LiDAR を用いた自己位置推定が幾何学的に縮退する局面が生じ得る.
このような環境特性を踏まえ,本研究では,センサ構成そのものの冗長化ではなく,
観測品質の変動を前提とした情報統合手法を採用する.

以下では,まず移動ロボットプラットフォームおよび搭載センサについて述べ,
次に LiDAR--IMU オドメトリ,ナビゲーションシステム,
およびGNSS品質監視に基づくソフトウェアアーキテクチャについて説明する.

\subsection{移動ロボットプラットフォームと搭載センサ}
\label{subsec:hardware}
移動ロボットには,不整地走行を想定したAGVシャーシを用い,
圃場内を巡回しながら各種センサデータを取得できる構成とした.
AGVシャシではGmade 社のGS02を採用した.
環境認識・自己位置推定のために全方位ソリッドステート式3D-LiDAR(Livox社Mid-360)を使用する.
LiDAR 点群は地図生成および自己位置推定の主要な観測情報として用いる.
LiDAR に内蔵された IMU は,
点群の運動歪み補正および短時間の姿勢推定に利用される.

絶対位置計測には2周波RTK-GNSS受信機(u-blox社ZED-F9P)を用い,
基準局から配信されるRTCM補正情報を利用することで,
屋外環境においてセンチメータ級の測位精度を目指す.
また,RTCM補正情報の取得にあたり,
森林総合研究所が公開しているRTK基準局サービスを利用した\cite{jforest_rtk}.
RTCM補正情報は無線LANを介してロボットに入力され,
GNSS受信機によりリアルタイムで適用される.

一方で,農業用ハウス周辺では,
作物や構造物による遮蔽およびマルチパスの影響により,
測位状態や共分散が時間的に大きく変動することが確認されている.
そのため,本研究では GNSS 観測を常に信頼可能な情報として扱うのではなく,
後述する品質監視に基づいて,
自己位置推定への寄与を動的に制御する.

また,ロボットの駆動にはBLDC(Brushless Direct Current)モータを使用するが,
本研究では不整地でのスリップが顕著であったため,
車輪オドメトリは自己位置推定の主要情報源として用いない.
センサデータ取得と推定処理はRaspberry Pi 4B上で実行し,無線LANを介して通信を行う.
AGVの写真や構成をFig.~\ref{fig:robot_platform}に示す.

\begin{figure}[t]
  \centering
  \includegraphics[width=\linewidth]{figures/3/AGV.pdf}
  \caption{移動ロボットプラットフォームと搭載センサ構成}
  \label{fig:robot_platform}
\end{figure}

\subsection{ソフトウェア構成}
\label{subsec:software}
本システムはROS~2(Robot Operating System 2)\cite{ref:ROS2} を基盤とし,自己位置推定・地図生成・自律走行をモジュール化して構成する.
自己位置推定およびマッピングは,フロントエンドとしてLiDAR--IMUオドメトリを
用いて高頻度の相対運動推定を行い,
バックエンドとしてRTAB-Map\cite{ref:2018rtabmap}によりループ閉合と因子グラフ最適化を実行する.
さらに,GNSS品質監視モジュール(GNSS Supervisor)により観測の採否および共分散調整を行ったRTK-GNSS観測を,
バックエンド最適化の位置制約として導入する.
自律走行はNavigation 2\cite{ref:Navigation2}フレームワークを用いて実装し,
地図およびロボット姿勢に基づく経路計画と追従制御を行う.

\subsubsection{ROS~2}
\label{subsubsec:ros2}
ソフトウェア基盤にはROS~2(Robot Operating System 2)\cite{ref:ROS2}を採用する.
ROS~2はDDSに基づく通信機構を備え,分散処理,実運用における信頼性,およびモジュール再利用性の観点から,
研究開発用ロボットシステムの構築に適している.本研究では,各センサのデータ取得,自己位置推定,地図生成,
および自律走行をノードとして分離し,トピックおよびTFにより統合する.

\subsubsection{Navigation2}
\label{subsubsec:navigation}
自律走行にはNavigation2を用いる.
Navigation2は,グローバル経路計画,ローカルプランニング,
Behavior Treeによるタスク実行を統合的に提供する.
本システムでは,バックエンドSLAMが提供する地図座標系における自己位置に基づいて経路計画を行い,
LiDAR観測に基づく局所障害物回避を併用する.
局所コストマップ生成にはSpatio-Temporal Voxel Layer(STVL)\cite{ref:stvl}を用い,
3次元点群を時系列に保持するボクセル表現により,近傍障害物の占有推定を行う.
これにより,農業環境における不整地や局所的な遮蔽物に対しても,
走行時の安全性と追従安定性の向上を図る.

\subsubsection{LiDAR--IMUオドメトリおよびSLAM}
\label{subsubsec:lio_slam}
フロントエンドにはFAST-LIO2\cite{ref:fast_lio2}を採用し,
LiDAR点群とIMU計測から高頻度の相対オドメトリを推定する.
FAST-LIO2は,高速なスキャンマッチングとIMU統合により,
リアルタイム動作に適したオドメトリを提供する.

バックエンドにはRTAB-Mapを用い,
ループ閉合検出と因子グラフ最適化により地図整合性を向上させる.
RTAB-Mapを採用した理由は,(i)ループ閉合に基づく長期ドリフト抑制を実装容易に実現できること,
(ii)複数センサ入力(LiDAR,GNSS等)を制約として扱える枠組みを有すること,
(iii)将来的に視覚センサ等を追加した場合でも入力モダリティの拡張が比較的容易であり,
農業環境マッピングへの発展性が高いこと,の3点である.

\subsubsection{座標系とTF構成}
\label{subsubsec:tf}
本システムの座標系は以下のように構成する.
\begin{itemize}
  \item FAST-LIO2:\texttt{odom} $\rightarrow$ \texttt{base\_link}(高頻度相対オドメトリ)
  \item RTAB-Map:\texttt{map} $\rightarrow$ \texttt{odom}(バックエンド最適化により更新される変換)
  \item 静的外部パラメータ:\texttt{base\_link} $\rightarrow$ \texttt{lidar},\texttt{base\_link} $\rightarrow$ \texttt{gps\_antenna}
\end{itemize}
これにより,制御・追従は\texttt{odom}系の連続性を保持しつつ,
地図参照のナビゲーションおよび環境マッピングは\texttt{map}系でのグローバル整合を利用できる.

\subsubsection{データ処理フロー}
\label{subsubsec:dataflow}
主なデータフローを以下に示す.
\begin{itemize}
  \item FAST-LIO2:点群・IMU $\rightarrow$ $\,^{\texttt{odom}}\!T_{\texttt{base}}(t)$
  \item GNSS Supervisor:\texttt{NavSatFix}+受信機状態量+LIOオドメトリ $\rightarrow$ 採否・共分散調整済みGNSS観測
  \item RTAB-Map:点群+LIOオドメトリ+品質監視後GNSS $\rightarrow$ ループ閉合/因子グラフ最適化 $\rightarrow$ \texttt{map}$\rightarrow$\texttt{odom},最適化軌跡
  \item Nav2:地図(\texttt{map})+自己位置(\texttt{map}系)+局所コストマップ(STVL) $\rightarrow$ 経路計画/追従制御
  \item 生育環境マッピング:最適化軌跡に基づき環境センサ値を\texttt{map}座標系へ投影
\end{itemize}

\clearpage
\section{農業用ハウス環境におけるGNSS観測異常の分類}
\label{sec:problem_definition}
農業用ハウス環境におけるRTK-GNSSは,多くの時間帯で高精度に利用可能である一方,
遮蔽・マルチパス・差分補正の途絶等により,測位品質が間欠的に劣化する.
本研究では,この品質劣化を単一の現象として扱わず,
オンラインで区別可能な故障モードとして次の3種類に整理する.

\begin{enumerate}
  \item \textbf{GNSS利用不可能}\\
  no-fix,使用衛星数不足,差分補正の途絶(RTCM ageの増大)等により,
  測位が成立しない,あるいは測定として利用できない状態である.これは測定欠落に相当する.
  \item \textbf{NLOS/マルチパス起因の大偏差外れ値混入}\\
  受信機がRTK-FIX等の状態を報告し,共分散も小さいにもかかわらず,
  遮蔽・マルチパスに起因するNLOS環境下で実際の測位が大きく偏り,
  推定軌跡に不連続な位置の飛び(位置ジャンプ)や断裂を生じる状態である.
  本研究の焦点は,この見かけ上は良好だが誤っている測位外れ値の抑制にある.
  \item \textbf{LIOの幾何学的退化(LiDARデジェネラシー)と不整地擾乱}\\
  ハウス内通路等では点群幾何が単調となり,可観測性が低下してLIOが退化し得る.
  また不整地では車輪スリップが顕著であり,実験では,Wheel Odometryは自己位置推定源としては不適切と判断した.
\end{enumerate}

本研究の目的は,GNSSが高精度に利用可能な時間帯が多い一方で,
遮蔽・マルチパス・補正途絶に起因して利用不可が間欠的に発生し,
まれにNLOS由来の大偏差外れ値が混入する統計的特性をもつ状況下で,
バックエンドのグラフ最適化を汚染しないことを最優先としつつ,
高品質RTK拘束によりLIOの長期ドリフトを抑制する統合推定系を構築することである.
そのために,後端へ注入するGNSS拘束をオンラインで選別する品質監視モジュールを設計する.


\section{提案手法:GNSS品質監視モジュールの設計}
\label{sec:supervisor}
本研究では,RTAB-Map に導入される GNSS 制約が因子グラフ最適化の結果を劣化させないことを最優先課題とし,
GNSS観測をオンラインで選別する GNSS 品質監視モジュールを設計した.
農業用ハウス環境において,RTK-GNSS は多くの時間帯で高精度に利用可能である一方,
遮蔽,マルチパス,および差分補正の途絶に起因して,NLOS環境下の大偏差外れ値が混入する可能性がある.
このとき,誤拘束が1サンプルでもRTAB-Mapへ注入されると,地図作成に劣化を引き起こす場合がある.
したがって本研究では,GNSS出力の重みを段階的に調整するのではなく,
採用または遮断によってRTAB-Mapへの拘束注入を制御する.

\subsection{入出力仕様と遮断の実装}
\label{subsec:supervisor_io}
品質監視モジュールの入力は,GNSS観測\texttt{NavSatFix},
受信機の補助状態量(測位状態,衛星数,DOP,補正受信状況など),
およびLIOが出力する相対オドメトリである.出力はRTAB-Mapへ与える\texttt{NavSatFix}であり,
観測の扱いを次の二値で決定する.

\begin{itemize}
  \item \textbf{Accept(採用)}: 観測をRTAB-Mapへ注入する(観測が持つ共分散を基本的に保持する).
  \item \textbf{Reject(遮断)}: 観測をRTAB-Mapへ注入しない,または固定の大きな対角共分散へ置換し,拘束として確実に無効化する.
\end{itemize}

NLOS/マルチパス環境下では,受信機が共分散を過小評価して報告する場合があり,
単純な倍率拡大による重み調整に依存すると誤拘束が残存し得る.
そこで本研究では遮断時に倍率拡大へ依存せず,
固定値の対角共分散により拘束を確実に無効化する実装とした.具体的には,遮断時の出力共分散を
$\mathrm{diag}(C_{\mathrm{blk}},C_{\mathrm{blk}},C_{\mathrm{blk}})$とし,
$C_{\mathrm{blk}}$は十分大きい定数(例:$99999$)として設定する.

\subsection{可用性判定}
\label{subsec:availability_gate}
可用性判定は,LIO等の外部推定に依存せず,
受信機自身が提供する状態量に基づいて GNSS観測の基本的な入力妥当性を判定する.
目的は,測位不成立や補正途絶などの利用不可能状態を確実に遮断し,
バックエンドグラフへ不正な拘束が注入されることを防ぐことである.

本研究では,補助状態量(UBX等)が一時的に欠落・遅延する状況を考慮し,
「補助情報が古い/未取得」であること自体では直ちに遮断せず,
明確なハード不成立条件のみを遮断トリガとした.
例として,
(1) 測位状態が規定未満,(2) 衛星数不足,(3) PDOP過大,
(4) 水平精度(hAcc)が閾値超過,(5) RTCM途絶が閾値超過,
などを用いる.

また,起動直後は履歴が十分でなく,増分整合性検定が成立しない場合があるため,
初期の一定サンプルは暫定的に採用として扱う初期化区間を設けた.
これにより正常区間が開始直後から連続して遮断になる挙動を抑制する.

また,起動直後は履歴が十分でなく増分整合性検定が成立しない場合があるため,
初期の一定サンプルはIntegrity gateを適用せず,
Availability gateのみで妥当性を確認した上で暫定的に採用する初期化区間を設けた.
これにより,正常区間であっても開始直後から連続して遮断となる挙動を抑制する.


\subsection{完全性監視:短時間増分整合性検定}
\label{subsec:integrity_gate}
完全性監視は,受信機状態が良好に見えるにもかかわらず,
遮蔽・マルチパスに起因するNLOS環境下で外れ値を含む誤測位が混入する状況を抑制するための機構である.
本研究では絶対位置の一致を仮定せず,短時間窓$\Delta T$の位置増分を比較する.
時刻$t$におけるGNSS位置(ENU)を$\bm{p}_{\mathrm{gnss}}(t)$,
LIOオドメトリを$\bm{p}_{\mathrm{lio}}(t)$とし,
増分残差を
\begin{equation}
  \bm{r}(t)=\bigl[\bm{p}_{\mathrm{gnss}}(t)-\bm{p}_{\mathrm{gnss}}(t-\Delta T)\bigr]
           -\bigl[\bm{p}_{\mathrm{lio}}(t)-\bm{p}_{\mathrm{lio}}(t-\Delta T)\bigr]
\end{equation}
で定義する.
GNSS共分散$\Sigma_{\mathrm{gnss}}(t)$は\texttt{NavSatFix}から抽出し,
LIO側は等方近似$\Sigma_{\mathrm{lio}}=\sigma_{\mathrm{lio}}^{2}\bm{I}$を用いる.
\begin{equation}
  \bm{S}(t)=\Sigma_{\mathrm{gnss}}(t)+\Sigma_{\mathrm{lio}}(t),
  \qquad
  d(t)=\bm{r}(t)^{\mathsf{T}}\bm{S}(t)^{-1}\bm{r}(t)
\end{equation}
を計算し,$d(t)$が閾値$\gamma$を超える場合にRejectとする.

ここで,NLOSによる過小共分散偽装に対して検定が過敏化しないよう,
$\Sigma_{\mathrm{gnss}}$の対角要素には下限値(covariance floor)を与える.
さらに,判定のチャタリングを避けるため,
$d(t)$に短い履歴の中央値フィルタを適用し,
連続不合格が一定回数に達した場合にRejectを維持する(ヒステリシス).

\subsection{外れ値発生直後の注入防止}
\label{subsec:first_frame_block}
増分整合性検定は窓幅や履歴処理を伴うため,原理的に検知遅延が生じ得る.
しかしNLOS由来の大偏差外れ値においては,
異常発生直後の最初の観測がバックエンド最適化へ注入されるだけで,
地図推定結果が大きく損なわれる場合がある.
そこで本研究では,外れ値発生直後の 1 サンプル目を確実に遮断することを目的として,
以下の二つの機構を組み合わせた.


(1) \textbf{即時遮断ルール}:
GNSS が極端に小さい共分散,すなわち過剰な信頼度を報告している場合に限り,
増分残差のノルム $\|\bm{r}(t)\|$ が微小閾値 $r_{\mathrm{tw}}$ を超えた瞬間に,
当該観測を遮断と判定する.
本規則は,小共分散かつ系統的偏りを伴うNLOS外れ値に対して,
同一サンプルでの即時遮断を実現することを目的としている.

(2) \textbf{1 サンプル遅延による決定後出力}: 
時刻 $t$ における GNSS 観測は即時に RTAB-Map へ渡さず,一旦バッファに保持する.
次サンプル $t+\delta$ において得られた判定結果が Reject であった場合には,
バッファされていた時刻 $t$ の観測を破棄する.
この処理により,窓検定に起因する検知遅延が存在する場合であっても,
外れ値発生直後の最初の観測がバックエンド最適化へ注入される確率を低減できる.
本研究では,地図破綻リスクの低減を最優先とし,
1 サンプル分のレイテンシ増加を許容した.

\section{完全性監視のための座標変換}
\label{sec:enu_conversion}
完全性監視では,GNSSとLIOの短時間増分を同一の直交座標系で比較する必要があるため,
GNSS測位(緯度・経度)を局所直交座標へ変換する.
GNSSが出力する測位は一般にWGS84(World Geodetic System 1984)に基づく地理座標
$(\varphi,\lambda,h)$(緯度・経度・楕円体高)で表される.
一方,完全性検定に用いる残差計算は,距離・増分を扱いやすいENU座標系
(East--North--Up,東・北・上の局所直交座標)で行うことが望ましい.

なお,本研究ではENU座標を品質監視モジュール内部の計算にのみ使用し,
RTAB-Mapへの入力は既存インタフェース互換性を優先して \texttt{NavSatFix} 形式のまま維持する.

GNSSの地理座標から局所ENUへの厳密変換には,
GeographicLib\cite{ref:GeographicLib}等による楕円体モデルに基づく座標変換が広く用いられる.
GeographicLibは高精度かつ汎用的であり,広範囲移動や標高変化を含む条件で特に有効である.
しかし本研究の対象はハウス周辺の狭い走行範囲であり,
さらに監視モジュールはRaspberry Pi 4B上で常時動作するため,
依存ライブラリ追加を避けた実装を優先した.
このため,完全性監視で必要な精度に対して十分と判断し,
以下の短距離平面近似によるWGS84$\rightarrow$ENU変換を採用した.

原点$(\varphi_0,\lambda_0)$は,初期の高品質測位(例:RTK-FIXが連続し,
可用性判定を満たす区間)から定め,
以降は同一原点に対して一貫して変換を行う.
地球半径$R$を用いて
\begin{align}
  x &\approx (\lambda-\lambda_0)\cos\varphi_0 \cdot R, \\
  y &\approx (\varphi-\varphi_0)\cdot R
\end{align}
により局所座標$(x,y)$を得る.
ここで$x$はEast,$y$はNorthに対応する.
本研究の完全性監視は増分に基づくため,
同一原点のもとで一貫して変換が行われる限り,数cm$\sim$数mスケールの局所運動に対して
近似誤差は支配的とならない.
ただし,走行範囲が大きくなる場合や高度差が無視できない場合には,
楕円体モデルに基づく厳密変換(GeographicLib等)の導入が望ましい.


\clearpage
\section{実験と評価}
\label{sec:exp_evaluation}
本節では,実環境(信州大学農場のハウス内通路)で収集したROS~2 bagデータを用い,
提案するGNSS品質監視(GNSS Supervisor)がRTAB-Mapの因子グラフ最適化に対する誤拘束注入リスクを低減し,
地図整合性を改善できることを検証する.
なお,本環境ではRTK-GNSSが安定して利用可能な時間帯が多い一方で,
遮蔽・マルチパスに起因する明確なNLOS外れ値は発生頻度が低く,再現性のある収集が困難であった.
そこで本研究では,NLOS由来の測位逸脱が疑われるROS 2 bag に基づく走行記録(以下,ログ)をケーススタディとして詳細に解析し,
品質監視の有効性を主として定性的に示す.

\subsection{実験環境とデータ収集}
\label{subsec:exp_setup}
実験は信州大学農場のハウス内通路で実施し,AGVを走行させながら
LiDAR点群,IMU,GNSS(\texttt{/fix}),および受信機状態量(例:測位状態,衛星数,DOP,RTCM age等)を
ROS~2 bagとして同期記録した.
収集後の解析は研究室環境でオフライン実行し,同一bagに対して各手法を適用した.
実験環境の概観をFig.~\ref{fig:env_photos}に示す.

% 環境写真(2枚並べ)
\begin{figure}[h]
  \centering
  \begin{subfigure}[t]{0.49\linewidth}
    \centering
    \includegraphics[width=\linewidth]{figures/3/environment_1.jpg}
    \caption{Experimental site overview (near the greenhouse field)}
    \label{fig:env_photo_outside}
  \end{subfigure}
  \hfill
  \begin{subfigure}[t]{0.49\linewidth}
    \centering
    \includegraphics[width=\linewidth]{figures/3/environment_2.jpg}
    \caption{Experimental site overview (inside the greenhouse)}
    \label{fig:env_photo_inside}
  \end{subfigure}
  \caption{Experimental environment}
  \label{fig:env_photos}
\end{figure}

\subsection{評価対象データにおけるGNSS軌跡不連続の観測}
\label{subsec:nlos_observation}
記録したGNSS観測(\texttt{/fix})をFoxglove上で可視化したところ,
本来連続であるべき走行軌跡に対し,短時間の不連続(ジャンプ/断裂)に相当する挙動が確認された.
Fig.~\ref{fig:foxglove_nlos}に,異常挙動が観測された2日分(2025-10-23,2025-10-20)の例を示す.
10/23のログでは走行途中で不自然な横方向の逸脱が現れており,本研究の評価対象ログとして用いる.
一方,10/20のログでも同様の挙動が観測されたが,走行終盤での発生であり,
以降の区間が短く最適化過程の比較に適さないため,本節では「同様の現象が複数日に渡って観測され得る」ことの補助的証拠として提示するに留める.
なお,本研究では,当該不連続が遮蔽・マルチパスに起因するNLOS外れ値(受信機状態が一見良好でも測位が大きく偏る事象)として現れ得る点に着目し,
当該区間をNLOS由来の測位逸脱が疑われる区間として解析対象に設定した.


\begin{figure}[t]
  \centering
  \begin{subfigure}[t]{0.49\linewidth}
    \centering
    \includegraphics[width=\linewidth]{figures/3/10_23.png}
    \caption{2025-10-23: overview}
    \label{fig:foxglove_1023_overview}
  \end{subfigure}
  \hfill
  \begin{subfigure}[t]{0.49\linewidth}
    \centering
    \includegraphics[width=\linewidth]{figures/3/10_23_0.png}
    \caption{2025-10-23: zoomed view around the anomaly}
    \label{fig:foxglove_1023_zoom}
  \end{subfigure}

  \vspace{2mm}

  \begin{subfigure}[t]{0.49\linewidth}
    \centering
    \includegraphics[width=\linewidth]{figures/3/10_20.png}
    \caption{2025-10-20: overview (reference only)}
    \label{fig:foxglove_1020_overview}
  \end{subfigure}
  \hfill
  \begin{subfigure}[t]{0.49\linewidth}
    \centering
    \includegraphics[width=\linewidth]{figures/3/10_20_0.png}
    \caption{2025-10-20: zoomed view around the anomaly}
    \label{fig:foxglove_1020_zoom}
  \end{subfigure}
  \caption{Examples of GNSS trajectory discontinuities visualized in Foxglove (topic: \texttt{/fix})}
  \label{fig:foxglove_nlos}
\end{figure}

\subsection{比較手法}
\label{subsec:methods_compared}
以下の3手法を比較する.いずれも同一のLiDAR/IMU入力を用い,RTAB-Mapの設定は可能な限り同一とした.
\begin{enumerate}
  \item \textbf{LIO only:} FAST-LIO2の相対オドメトリを用い,GNSS拘束を導入しない.
  \item \textbf{Naive fusion:} GNSS品質を考慮せず,\texttt{/fix}をそのままRTAB-Mapの位置制約として導入する.
  \item \textbf{Proposed:} 提案するGNSS Supervisorにより観測の採否(Accept/Reject)を判定し,
        採用観測のみをRTAB-Mapに入力する.
\end{enumerate}

\subsection{評価観点}
\label{subsec:metrics}
本ケーススタディでは外部計測機器による高精度Ground Truthが利用できないため,
地図整合性および推定の破綻有無を中心に定性的評価を行う.具体的には,
(i) 点群地図の幾何的一貫性(通路形状の崩れ,二重壁,ねじれ等),
(ii) 推定軌跡の連続性と不自然な折れ曲がり,
(iii) NLOS由来の測位逸脱が疑われる区間における推定の破綻・回復挙動,
を比較する.
また,提案法についてはReject判定のタイミングが異常区間と整合することを,
NISと判定の時系列により確認する(\S\ref{subsec:timeline_response}).

\subsection{実環境評価結果}
\label{subsec:real_world_results}
Figs.~\ref{fig:map_lio_only}--\ref{fig:map_naive_partial}に,
各手法で生成された点群地図と推定軌跡を示す(白:点群地図,黄:推定軌跡).
Naive fusionではNLOS由来の測位逸脱が疑われる区間において誤ったGNSS拘束が因子グラフへ注入され,
局所的な地図歪み(壁・通路形状の崩れ)が生じる場合が確認された.
一方,Proposedでは当該区間のGNSS観測が遮断され,
地図の一貫性が改善されることを確認した.


\clearpage

\begin{figure}[p]
  \centering

  % ---- Row 1 ----
  \begin{minipage}[t]{0.48\linewidth}
    \centering
    \includegraphics[width=\linewidth]{figures/3/Only_LIO.png}
    \captionof{figure}{Mapping result with LIO only (white: point cloud map, yellow: estimated trajectory).}
    \label{fig:map_lio_only}
  \end{minipage}
  \hfill
  \begin{minipage}[t]{0.48\linewidth}
    \centering
    \includegraphics[width=\linewidth]{figures/3/LIO_NLOS_FIX_all.png}
    \captionof{figure}{Mapping result with naive GNSS fusion (white: point cloud map, yellow: estimated trajectory).}
    \label{fig:map_naive_all}
  \end{minipage}

  \vspace{6mm}

  % ---- Row 2 ----
  \begin{minipage}[t]{0.48\linewidth}
    \centering
    \includegraphics[width=\linewidth]{figures/3/LIO+proposed.png}
    \captionof{figure}{Mapping result with the proposed GNSS Supervisor (white: point cloud map, yellow: estimated trajectory).}
    \label{fig:map_proposed_all}
  \end{minipage}
  \hfill
  \begin{minipage}[t]{0.48\linewidth}
    \centering
    \includegraphics[width=\linewidth]{figures/3/LIO_NLOS_FIX_part.png}
    \captionof{figure}{Example of local distortion observed after the anomaly in naive fusion (partial view).}
    \label{fig:map_naive_partial}
  \end{minipage}

\end{figure}

\clearpage


\subsection{NLOS疑い区間における挙動の詳細}
\label{subsec:NLOS_detail}
NLOS由来の外れ値の影響をより明確に示すため,Naive fusionにおいて推定が不安定化した区間の例を
Fig.~\ref{fig:map_naive_partial}に示す.
本ログでは,異常区間の直後に局所的な歪みが生じたものの,
後続区間で得られた整合的な観測およびループ閉合により,
全体としては回復する挙動も観測された.
しかし,この回復は常に保証されるものではなく,
異常の大きさや注入タイミングによっては地図破綻が回復不能となる可能性がある.
したがって,バックエンド汚染リスクを低減する観点からは,
NLOS由来の外れ値が疑われる観測を事前に遮断し,最適化へ投入しないことが重要である.

\subsection{品質監視の時間応答}
\label{subsec:timeline_response}
提案法が異常区間をどのように検知し遮断したかを示すため,
2025-10-23ログに対するNISと最終判定の時系列をFig.~\ref{fig:timeline_1023}に示す.
図中の破線は判定閾値$\gamma_{\mathrm{pass}}$を表し,上部のバーは最終的な採否(Accept/Reject)を示す.
異常区間ではNISが急増し,GNSS観測がRejectとして遮断されていることが分かる.
また,NISが閾値未満であっても,可用性判定や
回復ロックアウトによりRejectとなる場合がある.
本研究ではバックエンド最適化への誤拘束注入を最小化するため,
このような保守的な判定方針を採用した.

\begin{figure}[t]
  \centering
  \includegraphics[width=\linewidth]{figures/3/timeline_1023.pdf}
  \caption{Time history of the median NIS and the acceptance decision on the 2025-10-23 dataset. The dashed line indicates the threshold $\gamma_{\mathrm{pass}}$. The top bar shows the final PASS/FAIL decision (Accept/Reject). Note that GNSS updates can be rejected even when NIS is below the threshold due to availability checks (e.g., fix quality, auxiliary-data freshness) or recovery lockout, enforcing conservative integrity protection.}
  \label{fig:timeline_1023}
\end{figure}

\subsection{考察}
\label{subsec:discussion}
本ケーススタディにより,実環境においてNLOS由来の測位逸脱が疑われるGNSS観測が発生し得ること,
および品質監視により誤拘束の注入を抑制することで地図整合性が改善されることを確認した.
一方で,本環境ではNLOSの発生頻度が低く,統計的な再現実験を行うには追加の長時間収集が必要である.
今後は,複数ログに対する再現性評価,および外部基準(既知基準点,測量,あるいは高精度参照軌跡)を用いた
定量評価を課題とする.

\end{document}