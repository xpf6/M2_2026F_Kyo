%! TEX root = ../main.tex
\documentclass[main]{subfiles}

\begin{document}
\chapter{提案手法:GNSS品質監視に基づくロバスト自己位置推定}
Proposed Method: Robust Localization based on GNSS Quality Monitoring
写上大棚和照片
\section{問題の再定義:農業用ハウスにおけるGNSS信号品質の特性分析}
実測したデータ(良好時と不良時の/fixメッセージの協方差、衛星数、FIX/FLOAT状態の遷移など)を提示し、
ハウス環境におけるGNSS信号の不安定性と品質指標(協方差)の信頼性(正直さ)を定量的に示す。
TFツリーを示し、特にLiDARの傾斜搭載について言及する。

\section{ベースライン:LiDAR慣性オドメトリ (LIO)}
FAST-LIO2の概要と、本研究における適用方法(IMUキャリブレーション含む)を述べる。

LIO単独での精度限界(ドリフト)を示すための予備実験結果(あれば)を提示する。

\section{提案手法:GNSS品質監視モジュールと適応的EKF融合} 
C12880MAのデータシート要求(数MHz)に対し、
先行研究(割り込み方式)では数十~百数十kHzが限界であったことを示す(君のTable 1を引用)。


\subsection{全体アーキテクチャ} 
LIO、Wheel Odometry、GNSS Supervisor Module、EKF (robot localization) からなる融合システムのブロック図を示す。

\subsection{GNSS品質監視モジュール}
GNSSデータ(/fix)を入力とし、その品質(主に協方差行列、
必要ならFIX/FLOAT状態なども加味)を評価し、「信頼できるデータのみ」をEKFへ送る、
あるいは「品質情報を重みとして」EKFへ送るアルゴリズムを提案する(閾値処理や信頼度スケーリングなど)。
このモジュールが新規性の中核であることを強調する。

\subsection{EKFによる状態推定} 
Robot localization を用い、LIO (Wheel Odom) を
高頻度の主要な運動推定源とし、品質監視モジュールを経由した
GNSSデータを低頻度の絶対位置補正源として融合する設定を説明する。

\section{期待される効果}
GNSS信号が良いときはその精度を活用し、悪いときはLIOによってドリフトを抑制し、
全体として精度と頑健性を両立できることを論理的に説明する。

\end{document}