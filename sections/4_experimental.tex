%! TEX root = ../main.tex
\documentclass[main]{subfiles}

\begin{document}
\chapter{実験による評価 (Experimental Evaluation)}

\section{実験設定}
実験フィールド(信州大学農場ハウス)、使用したAGV、データ収集シナリオ
(GNSS良好時、不良時、ハウス内外移動時)を詳細に記述する。

評価に用いたGround Truthの定義(例:GNSS良好時の高精度軌跡、あるいは外部計測機器)を明確にする。

\section{定量的評価:自己位置推定精度の比較}
比較手法: (1) LIO (+Wheel Odom) のみ、(2) LIO+Wheel+GNSSのナイーブな融合(品質無視)、(3) 提案手法(品質監視付き融合)の3つを用意する。

各シナリオのデータセットに対して3手法を適用し、得られた軌跡をGround Truthと比較する。

結果の提示: (1) 軌跡比較図(Fig. 
%\ref{fig:trajectory_comparison}
)、
(2) 定量評価指標(RMSE, 最大誤差など)の比較表(Table 
%\ref{tab:localization_rmse}
)を示す。

\subsection{考察} 
実験結果に基づき、
提案手法(3)が比較手法(1)(2)に対して優位性を持つことを明確に論証する。
特に、GNSS品質が悪化した際に、ナイーブ融合(2)が破綻するのに対し、提案手法(3)が安定して精度を維持できることを強調する。

GNSS品質監視モジュールの閾値設定などのパラメータの妥当性についても議論する

\end{document}