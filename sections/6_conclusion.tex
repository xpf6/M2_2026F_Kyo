%! TEX root = ../main.tex
\documentclass[main]{subfiles}

\begin{document}
\chapter{結言}
\label{ch:conclusion}

\section{本研究のまとめ}

本研究では,農業用ハウス環境における巡回計測を想定し,
GNSS 測位品質が時間的に変動する条件下での自己位置推定の安定化と,
高速分光センシングを用いた生育環境情報の空間可視化について検討を行った.

まず,農業用ハウス環境における RTK-GNSS の挙動を整理し,
多くの時間帯では高精度な測位が得られる一方で,
遮蔽やマルチパスに起因する NLOS の影響により,
受信機状態量が良好である場合でも測位誤差が発生し得ることを,
実環境データに基づいて確認した.
このような外れ値がバックエンドの地図最適化処理に入力されることで,
地図整合性が損なわれる可能性がある点を課題として明確化した.

次に,上記課題に対する対処方策として,
LIOを短時間の参照とした
増分整合性に基づく GNSS 品質監視手法を構築した.
本手法は,GNSS 観測をバックエンドの因子グラフ最適化に直接入力するのではなく,
事前に拘束の採否を判定する構成を採用している.
実環境で取得した走行ログを用いた解析により,
NLOS の影響が疑われる区間において GNSS 拘束を抑制することで,
従来の手法と比較して地図の大きな歪みを回避できることを確認した.
ただし,本研究では検証例が限定的であり,
手法の一般性や定量的な性能評価については今後の検討課題である.

また,生育環境センシングの高密度化を目的として,
小型分光センサ C12880MA を対象とした高速駆動方式を検討した.
割込み駆動方式の制約を整理した上で,
ハードウェアトリガに同期した ADC/DMA ストリーミング取得方式を実装し,
STM32H723ZG 上において \SI{5}{\mega\hertz} の連続取得が可能であることを
実験により確認した.
さらに,ST 信号立下り後の遅延に起因する画素ずれを考慮した
有効画素範囲の補正を導入することで,
高速取得時におけるスペクトルデータの整合性を確保した.

加えて,上記の自己位置推定系およびセンシング系を統合し,
温度・湿度・CO$_2$ 濃度・気圧,
ならびに分光データに基づく特徴量を
空間情報と対応付けて可視化する処理系を構築した.
これにより,巡回計測によって取得される環境情報を
位置情報と関連付けて整理できることを示した.

なお,本研究では自律走行フレームワークとして
3D-LiDARによるナビゲーションシステムを導入し,
ノード構成および経路追従機能について,
研究室内環境において基本的な動作確認を行った.
しかし,圃場環境における自律走行実験については,
時間的制約のため十分な検証には至っておらず,
本研究では巡回計測システムとしての実装可能性を示すに留まっている.

以上より,本研究は,
農業用ハウス環境における GNSS 劣化の影響を考慮した
自己位置推定および環境センシング統合システムについて,
基礎的な設計と実装,ならびに限定的な実データに基づく検討を行ったものである.
特に,見かけ上良好なGNSS観測に混入し得るNLOS外れ値に対し,
バックエンド最適化への誤拘束注入を抑制するための実装指針を与えた点に意義がある.

\section{今後の課題}

今後の課題として,まず,本研究では外部基準軌跡を用いた厳密な誤差評価が行えていないため,
今後は既知基準点や測量データを用いた比較評価により,
検出性能や誤判定率を定量的に評価する必要がある.

また,本研究で用いた増分整合性に基づく判定手法についても,
LiDAR の幾何縮退や IMU 異常といった
他の誤差要因を考慮した拡張が今後の課題である.
複数の外れ値要因を統合的に扱うことで,
実環境における信頼性向上が期待される.

さらに,ナビゲーションシステムを含む自律走行系と完全に統合した
長時間巡回実験を実環境で実施し,
環境変化を含めた運用時の挙動を検証することが必要である.
これにより,本研究で構築した巡回計測基盤の
実運用に向けた課題がより明確になると考えられる.



\end{document}