%! TEX root = ../main.tex
\documentclass[main]{subfiles}

\begin{document}
\chapter{緒言}

\section{研究背景}
日本の農業では,農業従事者の減少と高齢化の進行により,
労働力不足が深刻な問題となっている\cite{ref:nourin}.
平成27年から令和5年にかけて,
基幹的農業従事者は175.7万人から116.4万人へと減少した.
特に65歳以上の従事者の割合が高く,
平均年齢も上昇している\cite{ref:roudou}.
この問題に対応するため,
日本政府は農業現場における農業データの利活用
の推進を支持している\cite{ref:nou_data}.
例えば,農業現場における土壌データ,
気象データなどの農業データを活用することで,
農作業の効率化やコストの削減を実現することができる.
そこで,我々は桑畑での適用を例として,
複数のAGV(Automatic Guided Vehicle,以下,AGV)を使用して圃場を巡回し,
桑の生育状態を自動で観察するシステムの構築を目指す.

\section{課題:農業環境における自律移動ロボットの課題}
AGVによる圃場モニタリング自動化の可能性を示す。
しかし、屋外不整地、特に農業用ハウスのような半構造化環境における自己位置推定の困難性を問題提起する。
不整地でのスリップによるオドメトリ誤差。
ハウス骨格によるGNSS信号の遮蔽・マルチパス、それに伴う信号品質の動的な変動(君の実験結果をここで軽く触れる)。

\section{先行研究とその限界}
一般的な屋外SLAM/LIO技術(FAST-LIO2等)を紹介し、ドリフト蓄積の問題点を指摘する。
GNSSを用いたセンサーフュージョン技術を紹介するが、GNSS信号が常に良好であることを前提としている研究が多い点を指摘し、ハウス環境への適用限界を示す。
本研究室の先行研究(小林さん)に触れ、センサデータ収集は行われたが、位置精度やロバストな自律走行には課題が残っていたことを明確にする。

\section{本研究の目的と新規性}
目的: GNSS信号品質が動的に変動する農業用ハウス環境において、
3D-LiDAR, IMU, Wheel Odometry, および品質情報付きGNSSをインテリジェントに統合し、
途切れなく(continuous)、信頼性の高い(high-integrity)3D自己位置推定を実現する手法を開発すること。
さらに、その高精度な軌跡に基づき、高分解能な生態環境マップを生成すること。

新規性: 
(1) ハウス環境特有のGNSS品質変動パターンを実測に基づき分析・モデル化する点。
(2) GNSS品質情報(協方差等)に応じて融合システムへの寄与を動的に調整する
「GNSS品質監視・融合制御」アルゴリズムを提案・実装する点。(3) 上記技術の有効性を実環境データで定量的に実証する点。

\section{本論文の構成}
各章の概要を簡潔に述べる。 

\end{document}