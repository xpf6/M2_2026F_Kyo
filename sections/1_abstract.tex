%! TEX root = ../main.tex
\documentclass[main]{subfiles}

\begin{document}
\chapter{緒言}

\section{研究背景}
日本の農業分野では,基幹的農業従事者の減少および高齢化の進行に伴い,
労働力不足が深刻な社会課題となっている\cite{ref:nourin,ref:roudou}.
実際に,平成27年から令和5年にかけて基幹的農業従事者数は175.7万人から116.4万人へ減少しており,
高齢層の比率増加と平均年齢の上昇も報告されている\cite{ref:roudou}.
この課題に対し,農業現場で取得される各種データの利活用を通じた生産性向上・省力化の推進が
政策的にも進められている\cite{ref:nou_data}.

近年は,土壌・気象・生育・作業履歴などの農業データを継続的に収集し,
圃場状態を定量的に把握するデータ駆動型の営農が注目されている.
これを実現するうえで,広大な圃場を自律的に巡回し,人手に頼らずモニタリングを行う
自律移動ロボット(Automatic Guided Vehicle,以下,AGV)による圃場巡回モニタリングは有力な手段である.
本研究では,桑畑での適用を一例として,AGVを用いて圃場を巡回し,
環境情報および生育に関連する計測情報を自動取得・可視化するシステムの構築を目指す.


\section{農業環境における課題}
自律移動ロボットの実用化において,農業用ビニールハウスのような環境は,
一般的な屋外環境や屋内環境とは異なる独自の課題を有する半構造化環境\cite{agri_slam_2023}である.
第一に,ハウスの骨組みや被覆材,周囲の作物によりGNSS(Global Navigation Satellite System,以下GNSS)信号の
遮蔽やマルチパスが頻発するため,安定した衛星測位が困難である.
第二に,ハウス内の通路は狭隘かつ単調な形状であり,
3D-LiDARを用いたSLAM(Simultaneous Localization and Mapping,以下SLAM)において,
幾何学的特徴の欠落による自己位置推定の破綻が生じやすい.
これらの要因により,農業環境では単一のセンサに依存した自己位置推定は信頼性に欠ける傾向にある.

\section{本研究の目的}
本研究の目的は,GNSS測位品質が動的に変動する農業用ハウス環境において,
3D-LiDAR,IMU(Inertial Measurement Unit,以下IMU),および品質情報を伴うGNSSを統合し,
高信頼性を有する3次元自己位置推定を実現する手法を開発することである.
さらに,得られた高信頼な軌跡に基づき,
温湿度,CO$_2$濃度,気圧,および分光センサによるスペクトル情報などの
生育環境情報を圃場空間へ高精度に投影し,
作物の生育管理に資する高分解能な生育環境マップを生成することを目的とする.

\section{本論文の構成要素}
本論文は全6章から構成される.
第2章では,農業ロボットにおける自己位置推定および環境センシングに関する関連研究と本研究の位置づけを述べる.
第3章では,GNSSの品質監視(Supervisor)に基づくロバストな自己位置推定手法の提案と評価について述べる.
第4章では,高速かつ低遅延なデータ取得を実現する分光センシングシステムのハードウェア構築および駆動手法について述べる.
第5章では,提案した位置推定手法とセンシングシステムを統合した実環境実験を行い,
生成された生育環境空間マップの整合性と有効性を論じる.
第6章では,本研究の結論と今後の展望を述べる.

なお,本システムは遠隔監視のための無線通信機能を有するが,
本論文では自律移動の基盤となるロバストな自己位置推定および高速センシング技術の確立に焦点を当てるため,
通信アーキテクチャの詳細な実装や評価については本論文の範囲外とする.

すなわち,本研究は「測位品質が変動する農業環境においても破綻しない自己位置推定」を基盤として,
巡回センシングによる圃場全体の高分解能マップ化を実現する統合システムの確立を目指す.


\end{document}