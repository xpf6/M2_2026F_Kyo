%! TEX root = ../main.tex
\documentclass[main]{subfiles}

\begin{document}
\chapter{緒言}

\section{研究背景}
日本の農業分野では,担い手の減少および高齢化の進行に伴い,労働力不足が深刻な社会課題となっている.
農林水産省の令和6年度白書によれば,基幹的農業従事者数は約20年間で半減し,
平成12年の240万人から令和6年には111万4千人へ減少している.
また,65歳以上が全体の71.7\%を占め,平均年齢は69.2歳である\cite{ref:maff_whitepaper_r6_section3}.
Fig.~\ref{fig:maff_core_farmers}に,基幹的農業従事者数と平均年齢の推移を示す.

このような労働制約の下で,生産性向上と省力化を同時に実現する手段として,
農業現場で取得される多様なデータを利活用するデータ駆動型の営農が重要となる.
これを実現するうえで,広い圃場を反復的に巡回し,人手に頼らずモニタリングを行うための手段として,
自律移動ロボット(Automatic Guided Vehicle,以下,AGV)による圃場巡回モニタリングは有効である.
本研究では,桑畑での適用を一例として,AGVを用いて圃場を巡回し,
環境情報および生育に関連する計測情報を自動取得・可視化するシステムの構築を目指す.

\begin{figure}[t]
  \centering
  \includegraphics[width=0.92\linewidth]{figures/1/1.png}
  \caption{Number of core agricultural workers and average age.\cite{ref:maff_whitepaper_r6_section3}}
  \label{fig:maff_core_farmers}
\end{figure}

近年,スマート農業の推進により,センシング,情報通信,位置情報基盤,データ管理,
および自動走行農機・ロボット等の技術要素を統合し,
営農の省力化・高度化を図る取り組みが進められている\cite{ref:maff_smart_meguji_2025}.
スマート農業の実装においては,圃場内の状態を継続的に計測し,
空間情報として蓄積・可視化する計測基盤が不可欠である.
AGVに搭載したセンサによる巡回計測は,走行に伴い広い範囲を連続的に観測できるため,
空間密度と更新頻度を両立しやすい.
さらに,自動走行により人手依存を低減しつつ,反復的な観測を可能にする.
また,スマート農業では,RTK-GNSS等の位置情報基盤,圃場周辺作業の省力化,
情報通信環境整備,センシング・データ管理が相互に接続される\cite{ref:maff_smart_meguji_2025}.
Fig.~\ref{fig:maff_smart_concept}に,スマート農業の施策イメージを示す.

\begin{figure}[t]
  \centering
  \includegraphics[width=0.98\linewidth]{figures/1/2.png}
  \caption{Conceptual overview of smart agriculture initiatives.\cite{ref:maff_smart_meguji_2025}}
  \label{fig:maff_smart_concept}
\end{figure}

\section{農業環境における課題}
\label{sec:intro_challenges}
巡回計測を担う移動ロボットの実用化には,計測データに取得位置を正確に付与し,
圃場座標上に空間的に配置するための高信頼な自己位置推定が前提となる.
しかし,農業用ビニールハウスのような半構造環境\cite{tanaka2013semi}では,
屋外・屋内のいずれとも異なる複合的困難が存在する.

第一に,ハウス骨組み,被覆材,周辺植生等によりGNSS信号が遮蔽・マルチパスを受けやすく,
RTK測位においてもNLOS(Non-Line-of-Sight,以下,NLOS)に起因する偏り外れ値が生じ得る.
この種の外れ値がバックエンド最適化へ誤拘束として注入されると,地図整合性を破壊し,
地図歪みを引き起こす恐れがある.

第二に,ハウス内通路は狭隘かつ単調な幾何形状となりやすく,
LiDARスキャンマッチングにおける幾何学的縮退が生じやすい.
その結果,LiDAR--IMU Odometry(LIO)は短時間では高精度であっても,
長時間走行でドリフトが蓄積し,絶対位置基準との整合が課題となる.

第三に,農業計測はマルチモーダル化が進み,温湿度やCO$_2$などの環境量に加え,
分光情報のような高次元データを用いた生育評価も重要となっている.
移動体に搭載する場合,取得周期・遅延・同期精度が空間マップ品質を支配するため,
計測と位置推定が同時に要求される.
したがって,農業用ハウス環境での巡回センシングを意思決定へ接続するには,
劣化し得るGNSSを前提としたロバストな統合測位と高頻度センシングの空間マッピングを
統合的に設計する必要がある.

\section{本研究の目的}
\label{sec:intro_objective}
本研究の目的は,GNSS測位品質が動的に変動する農業用ハウス環境において,
3D-LiDAR,IMU,および品質情報を伴うRTK-GNSSを統合し,
バックエンド最適化の破綻を回避しながら高信頼な自己位置推定を実現することである.
特に,受信機状態量のみでは判別が難しいNLOS外れ値に着目し,
LIOの短時間増分との整合性に基づきGNSS観測の採否・重み付けをオンラインに制御する
GNSS品質監視モジュールを提案する.

さらに,得られた高信頼な軌跡に基づき,
温湿度,CO$_2$濃度,気圧等の環境センサ値を取得し,小型分光センサによるスペクトル情報および位置情報と統合し,
生育環境の高分解能マップを生成することを目的とする.
これにより,人手による観測頻度・空間密度の制約を緩和し,
圃場状態の定量把握と管理作業の高度化に資する巡回計測を提示する.

\section{本論文の構成}
\label{sec:intro_outline}
本論文は全6章から構成される.
第2章では,農業ロボットにおける自己位置推定および環境センシングに関する関連研究を整理し,
本研究の位置づけを示す.
第3章では,GNSS品質監視に基づくロバストな自己位置推定手法の設計と評価を述べる.
第4章では,高速なデータ取得を実現する分光センシングシステムのハードウェア構築と駆動方式を述べる.
第5章では,提案した位置推定手法とセンシングシステムを統合し,実環境における生育環境マップ生成を示す.
第6章では,本研究の結論と今後の展望を述べる.

\end{document}
