%! TEX root = ../main.tex
\documentclass[main]{subfiles}

\begin{document}
\chapter{緒言}

\section{研究背景}
日本の農業分野では,基幹的農業従事者の減少および高齢化の進行に伴い,
労働力不足が深刻な社会課題となっている\cite{ref:nourin,ref:roudou}.
実際に,平成27年から令和5年にかけて基幹的農業従事者数は175.7万人から116.4万人へ減少しており,
高齢層の比率増加と平均年齢の上昇も報告されている\cite{ref:roudou}.
この課題に対し,農業現場で取得される各種データの利活用を通じた生産性向上・省力化の推進が
政策的にも進められている\cite{ref:nou_data}.

近年は,土壌・気象・生育・作業履歴などの農業データを継続的に収集し,
圃場状態を定量的に把握するデータ駆動型の営農が注目されている.
これを実現するうえで,自律移動ロボットによる圃場巡回モニタリングは有力な手段である.
本研究では,桑畑での適用を一例として,AGV(Automatic Guided Vehicle,以下,AGV)を用いて圃場を巡回し,
環境情報および生育に関連する計測情報を自動取得・可視化するシステムの構築を目指す.

\section{課題:農業環境における自律移動ロボットの自己位置推定}
AGVによる圃場モニタリング自動化の可能性を示す一方で,
屋外不整地,特に農業用ハウスのような半構造化環境における
自己位置推定の困難性を問題提起する.
不整地でのスリップによるオドメトリ誤差.
ハウス骨格によるGNSS信号の遮蔽・マルチパス,それに伴う信号品質の動的な変動.

圃場モニタリングを自動化するためには,
ロボットが圃場での自律走行や取得したセンサデータを空間座標に正確に付与する必要がある.
しかし,農業環境,特に農業用ハウスのような半構造化環境においては,
自己位置推定の成立条件が容易に損なわれる.主な要因は以下のとおりである.

第一に,屋外不整地では,路面の凹凸,ぬかるみ,段差等により車輪スリップや沈下が発生しやすく,
Wheel Odometry に基づく推定誤差が蓄積する.
第二に,ハウス骨格(鉄骨フレーム)や被覆材,周辺構造物に起因して
GNSS信号の遮蔽・反射(マルチパス)が生じ,測位品質が時間的に大きく変動する.
このため,GNSSの測位結果が連続的に利用できない区間や,
外れ値が断続的に混入する区間が発生し得る.
第三に,ハウス内の通路(長廊)のように幾何学的特徴が単調な環境や,
ハウス間の移動時における特徴点群が少ない環境では,
3D-LiDARに基づくスキャンマッチングの可観測性が低下して推定が退化し,ドリフトが増大する.

以上より,農業用ハウス環境では,GNSSとLiDARの双方が同時あるいは交互に劣化し得るため,
単純なセンサ統合では推定の破綻や跳変が生じ,取得データの空間整合性(マッピング信頼性)を損なう.
したがって,測位品質の変動を前提として,センサの信頼度に応じて融合系の寄与を動的に制御する枠組みが必要である.


\section{先行研究とその限界}
近年,3D-LiDARとIMUを用いたLiDAR--Inertial Odometry(LIO)が盛んに研究されており,
FAST-LIO2に代表される手法は,高頻度な状態推定と高い計算効率を両立しつつ,
屋外環境での自己位置推定に広く用いられている.
一方で,LIOは相対推定であるため,長時間走行ではドリフトが不可避であり,
環境の幾何的退化(長廊・反復構造・観測方向の偏り等)により推定誤差が急増することがある.

この課題に対し,GNSSを絶対位置制約としてLIOと融合する研究が多数報告されている.
しかし,多くの手法はGNSSが安定して利用可能であること,
あるいは測位誤差が概ねガウス分布に従うことを暗黙的に仮定し,
測位品質が急変する状況(遮蔽・マルチパス・Fix/Float切替等)への対処が十分に体系化されていない.農業用ハウス環境では,GNSSの測位品質が時変かつ断続的に劣化するため,固定的な測定雑音設定や単純な外れ値除去のみでは,推定の跳変や不整合を抑制できない可能性が高い.

また,本研究室の先行研究では,環境センサおよび分光センサを統合した計測ユニットをROS上で運用し,
圃場の定点におけるデータ収集が実施された.しかし,圃場全域を対象とした空間マッピングや,
自律移動を前提としたロバストな自己位置推定の観点では,GNSS品質変動下での推定安定化,
およびセンサデータの空間付与精度の検証が課題として残されていた.

\section{本研究の目的と新規性}
本研究の目的は,GNSS測位品質が動的に変動する農業用ハウス環境において,
3D-LiDAR,IMU,Wheel Odometry,および品質情報を伴うGNSSを統合し,
高連続性かつ高完全性(High Integrity)を有する3次元自己位置推定を
実現する手法を開発することである.さらに,得られた高信頼な軌跡に基づき,
温湿度,CO$_2$濃度,気圧,および分光ピーク波長などの生態環境情報を圃場空間へ投影し,
高分解能な生態環境マップを生成することを目的とする.

本研究の新規性は以下の3点に整理できる.
\begin{enumerate}
  \item 農業用ハウス環境におけるGNSS測位品質の時変変動(例:Fix/Float切替,衛星数低下,外れ値混入)を,実測データに基づき分析し,融合設計に必要な特性として整理する点.
  \item GNSS品質情報(協分散等)および時系列の整合性検定に基づき,融合系への寄与を動的に制御するGNSS品質監視・融合制御(Supervisor)を提案し,ヒステリシスを伴う状態遷移を含めて実装する点.
  \item 提案手法を実環境データで評価し,軌跡の跳変抑制,ドリフト低減,および生態環境マップの空間整合性向上を定量的に示す点.
\end{enumerate}

\section{本論文の構成}
本論文は全7章から構成される.第2章ではロボットプラットフォームおよび各種センサ構成,
ならびにデータ取得・時刻同期の方針を述べる.
第3章では農業用ハウス環境におけるGNSSおよびLiDARの退化要因を整理し,
本研究で扱う問題設定と評価指標(可用性・完全性等)を定義する.
第4章では退化感知に基づくGNSS品質監視・融合制御と,EKFを用いた統合推定フレームワークを提案する.
第5章では実験条件,比較手法,評価指標に基づく定量評価結果を示し,考察を行う.
第6章では得られた軌跡を用いた生態環境空間マッピングの結果を示し,
位置推定誤差がマップ信頼性へ与える影響を論じる.第7章では本研究の結論と今後の課題を述べる.


\end{document}