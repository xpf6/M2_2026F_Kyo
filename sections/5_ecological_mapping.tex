%! TEX root = ../main.tex
\documentclass[main]{subfiles}

\begin{document}
\chapter{応用:生態環境マッピング (Application: Ecological Mapping)}

\section{マッピング手法}
第4章で最も精度が高いと評価された提案手法による自己位置推定軌跡を用いることを明記する。

第2章で開発した高速生態センサユニットからのデータ(スペクトル、CO2、温湿度)を、
高精度なタイムスタンプと位置情報に同期させて3Dマップ上に重畳(オーバーレイ)する手法を
説明する(必要であれば補間手法なども)。

\section{生成された生態環境マップ}
実際に生成した各種生態環境マップ(スペクトル指標マップ、CO2濃度分布マップ、温湿度マップなど)を
提示する
%(Fig. \ref{fig:spectral_map}, Fig. \ref{fig:co2_map} など)。

\subsection{マッピング結果の考察} 
生成されたマップから読み取れる空間的な分布パターンや、
異なる環境要因間の相関(例:日照条件とスペクトル、CO2濃度と植物活性など)について考察する。

高精度な自己位置推定が、意味のある生態環境マップ生成にいかに貢献するかを具体的に示す

\end{document}